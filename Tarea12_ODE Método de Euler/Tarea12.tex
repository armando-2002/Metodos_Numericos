% Options for packages loaded elsewhere
\PassOptionsToPackage{unicode}{hyperref}
\PassOptionsToPackage{hyphens}{url}
\PassOptionsToPackage{dvipsnames,svgnames,x11names}{xcolor}
%
\documentclass[
  letterpaper,
  DIV=11,
  numbers=noendperiod]{scrartcl}

\usepackage{amsmath,amssymb}
\usepackage{iftex}
\ifPDFTeX
  \usepackage[T1]{fontenc}
  \usepackage[utf8]{inputenc}
  \usepackage{textcomp} % provide euro and other symbols
\else % if luatex or xetex
  \usepackage{unicode-math}
  \defaultfontfeatures{Scale=MatchLowercase}
  \defaultfontfeatures[\rmfamily]{Ligatures=TeX,Scale=1}
\fi
\usepackage{lmodern}
\ifPDFTeX\else  
    % xetex/luatex font selection
\fi
% Use upquote if available, for straight quotes in verbatim environments
\IfFileExists{upquote.sty}{\usepackage{upquote}}{}
\IfFileExists{microtype.sty}{% use microtype if available
  \usepackage[]{microtype}
  \UseMicrotypeSet[protrusion]{basicmath} % disable protrusion for tt fonts
}{}
\makeatletter
\@ifundefined{KOMAClassName}{% if non-KOMA class
  \IfFileExists{parskip.sty}{%
    \usepackage{parskip}
  }{% else
    \setlength{\parindent}{0pt}
    \setlength{\parskip}{6pt plus 2pt minus 1pt}}
}{% if KOMA class
  \KOMAoptions{parskip=half}}
\makeatother
\usepackage{xcolor}
\setlength{\emergencystretch}{3em} % prevent overfull lines
\setcounter{secnumdepth}{-\maxdimen} % remove section numbering
% Make \paragraph and \subparagraph free-standing
\makeatletter
\ifx\paragraph\undefined\else
  \let\oldparagraph\paragraph
  \renewcommand{\paragraph}{
    \@ifstar
      \xxxParagraphStar
      \xxxParagraphNoStar
  }
  \newcommand{\xxxParagraphStar}[1]{\oldparagraph*{#1}\mbox{}}
  \newcommand{\xxxParagraphNoStar}[1]{\oldparagraph{#1}\mbox{}}
\fi
\ifx\subparagraph\undefined\else
  \let\oldsubparagraph\subparagraph
  \renewcommand{\subparagraph}{
    \@ifstar
      \xxxSubParagraphStar
      \xxxSubParagraphNoStar
  }
  \newcommand{\xxxSubParagraphStar}[1]{\oldsubparagraph*{#1}\mbox{}}
  \newcommand{\xxxSubParagraphNoStar}[1]{\oldsubparagraph{#1}\mbox{}}
\fi
\makeatother

\usepackage{color}
\usepackage{fancyvrb}
\newcommand{\VerbBar}{|}
\newcommand{\VERB}{\Verb[commandchars=\\\{\}]}
\DefineVerbatimEnvironment{Highlighting}{Verbatim}{commandchars=\\\{\}}
% Add ',fontsize=\small' for more characters per line
\usepackage{framed}
\definecolor{shadecolor}{RGB}{241,243,245}
\newenvironment{Shaded}{\begin{snugshade}}{\end{snugshade}}
\newcommand{\AlertTok}[1]{\textcolor[rgb]{0.68,0.00,0.00}{#1}}
\newcommand{\AnnotationTok}[1]{\textcolor[rgb]{0.37,0.37,0.37}{#1}}
\newcommand{\AttributeTok}[1]{\textcolor[rgb]{0.40,0.45,0.13}{#1}}
\newcommand{\BaseNTok}[1]{\textcolor[rgb]{0.68,0.00,0.00}{#1}}
\newcommand{\BuiltInTok}[1]{\textcolor[rgb]{0.00,0.23,0.31}{#1}}
\newcommand{\CharTok}[1]{\textcolor[rgb]{0.13,0.47,0.30}{#1}}
\newcommand{\CommentTok}[1]{\textcolor[rgb]{0.37,0.37,0.37}{#1}}
\newcommand{\CommentVarTok}[1]{\textcolor[rgb]{0.37,0.37,0.37}{\textit{#1}}}
\newcommand{\ConstantTok}[1]{\textcolor[rgb]{0.56,0.35,0.01}{#1}}
\newcommand{\ControlFlowTok}[1]{\textcolor[rgb]{0.00,0.23,0.31}{\textbf{#1}}}
\newcommand{\DataTypeTok}[1]{\textcolor[rgb]{0.68,0.00,0.00}{#1}}
\newcommand{\DecValTok}[1]{\textcolor[rgb]{0.68,0.00,0.00}{#1}}
\newcommand{\DocumentationTok}[1]{\textcolor[rgb]{0.37,0.37,0.37}{\textit{#1}}}
\newcommand{\ErrorTok}[1]{\textcolor[rgb]{0.68,0.00,0.00}{#1}}
\newcommand{\ExtensionTok}[1]{\textcolor[rgb]{0.00,0.23,0.31}{#1}}
\newcommand{\FloatTok}[1]{\textcolor[rgb]{0.68,0.00,0.00}{#1}}
\newcommand{\FunctionTok}[1]{\textcolor[rgb]{0.28,0.35,0.67}{#1}}
\newcommand{\ImportTok}[1]{\textcolor[rgb]{0.00,0.46,0.62}{#1}}
\newcommand{\InformationTok}[1]{\textcolor[rgb]{0.37,0.37,0.37}{#1}}
\newcommand{\KeywordTok}[1]{\textcolor[rgb]{0.00,0.23,0.31}{\textbf{#1}}}
\newcommand{\NormalTok}[1]{\textcolor[rgb]{0.00,0.23,0.31}{#1}}
\newcommand{\OperatorTok}[1]{\textcolor[rgb]{0.37,0.37,0.37}{#1}}
\newcommand{\OtherTok}[1]{\textcolor[rgb]{0.00,0.23,0.31}{#1}}
\newcommand{\PreprocessorTok}[1]{\textcolor[rgb]{0.68,0.00,0.00}{#1}}
\newcommand{\RegionMarkerTok}[1]{\textcolor[rgb]{0.00,0.23,0.31}{#1}}
\newcommand{\SpecialCharTok}[1]{\textcolor[rgb]{0.37,0.37,0.37}{#1}}
\newcommand{\SpecialStringTok}[1]{\textcolor[rgb]{0.13,0.47,0.30}{#1}}
\newcommand{\StringTok}[1]{\textcolor[rgb]{0.13,0.47,0.30}{#1}}
\newcommand{\VariableTok}[1]{\textcolor[rgb]{0.07,0.07,0.07}{#1}}
\newcommand{\VerbatimStringTok}[1]{\textcolor[rgb]{0.13,0.47,0.30}{#1}}
\newcommand{\WarningTok}[1]{\textcolor[rgb]{0.37,0.37,0.37}{\textit{#1}}}

\providecommand{\tightlist}{%
  \setlength{\itemsep}{0pt}\setlength{\parskip}{0pt}}\usepackage{longtable,booktabs,array}
\usepackage{calc} % for calculating minipage widths
% Correct order of tables after \paragraph or \subparagraph
\usepackage{etoolbox}
\makeatletter
\patchcmd\longtable{\par}{\if@noskipsec\mbox{}\fi\par}{}{}
\makeatother
% Allow footnotes in longtable head/foot
\IfFileExists{footnotehyper.sty}{\usepackage{footnotehyper}}{\usepackage{footnote}}
\makesavenoteenv{longtable}
\usepackage{graphicx}
\makeatletter
\def\maxwidth{\ifdim\Gin@nat@width>\linewidth\linewidth\else\Gin@nat@width\fi}
\def\maxheight{\ifdim\Gin@nat@height>\textheight\textheight\else\Gin@nat@height\fi}
\makeatother
% Scale images if necessary, so that they will not overflow the page
% margins by default, and it is still possible to overwrite the defaults
% using explicit options in \includegraphics[width, height, ...]{}
\setkeys{Gin}{width=\maxwidth,height=\maxheight,keepaspectratio}
% Set default figure placement to htbp
\makeatletter
\def\fps@figure{htbp}
\makeatother

\KOMAoption{captions}{tableheading}
\makeatletter
\@ifpackageloaded{caption}{}{\usepackage{caption}}
\AtBeginDocument{%
\ifdefined\contentsname
  \renewcommand*\contentsname{Tabla de contenidos}
\else
  \newcommand\contentsname{Tabla de contenidos}
\fi
\ifdefined\listfigurename
  \renewcommand*\listfigurename{Listado de Figuras}
\else
  \newcommand\listfigurename{Listado de Figuras}
\fi
\ifdefined\listtablename
  \renewcommand*\listtablename{Listado de Tablas}
\else
  \newcommand\listtablename{Listado de Tablas}
\fi
\ifdefined\figurename
  \renewcommand*\figurename{Figura}
\else
  \newcommand\figurename{Figura}
\fi
\ifdefined\tablename
  \renewcommand*\tablename{Tabla}
\else
  \newcommand\tablename{Tabla}
\fi
}
\@ifpackageloaded{float}{}{\usepackage{float}}
\floatstyle{ruled}
\@ifundefined{c@chapter}{\newfloat{codelisting}{h}{lop}}{\newfloat{codelisting}{h}{lop}[chapter]}
\floatname{codelisting}{Listado}
\newcommand*\listoflistings{\listof{codelisting}{Listado de Listados}}
\makeatother
\makeatletter
\makeatother
\makeatletter
\@ifpackageloaded{caption}{}{\usepackage{caption}}
\@ifpackageloaded{subcaption}{}{\usepackage{subcaption}}
\makeatother

\ifLuaTeX
\usepackage[bidi=basic]{babel}
\else
\usepackage[bidi=default]{babel}
\fi
\babelprovide[main,import]{spanish}
% get rid of language-specific shorthands (see #6817):
\let\LanguageShortHands\languageshorthands
\def\languageshorthands#1{}
\ifLuaTeX
  \usepackage{selnolig}  % disable illegal ligatures
\fi
\usepackage{bookmark}

\IfFileExists{xurl.sty}{\usepackage{xurl}}{} % add URL line breaks if available
\urlstyle{same} % disable monospaced font for URLs
\hypersetup{
  pdftitle={Métodos Numericos},
  pdfauthor={José Sarango},
  pdflang={es},
  colorlinks=true,
  linkcolor={blue},
  filecolor={Maroon},
  citecolor={Blue},
  urlcolor={Blue},
  pdfcreator={LaTeX via pandoc}}


\title{Métodos Numericos}
\usepackage{etoolbox}
\makeatletter
\providecommand{\subtitle}[1]{% add subtitle to \maketitle
  \apptocmd{\@title}{\par {\large #1 \par}}{}{}
}
\makeatother
\subtitle{ODE Método de Euler}
\author{José Sarango}
\date{}

\begin{document}
\maketitle

\renewcommand*\contentsname{Tabla de Contenidos}
{
\hypersetup{linkcolor=}
\setcounter{tocdepth}{3}
\tableofcontents
}

\section{Conjunto de ejercicios}\label{conjunto-de-ejercicios}

\subsection{1. Use el método de Euler para aproximar las soluciones para
cada uno de los siguientes problemas de valor
inicial.}\label{use-el-muxe9todo-de-euler-para-aproximar-las-soluciones-para-cada-uno-de-los-siguientes-problemas-de-valor-inicial.}

\begin{Shaded}
\begin{Highlighting}[]
\ImportTok{import}\NormalTok{ logging}
\ImportTok{from}\NormalTok{ sys }\ImportTok{import}\NormalTok{ stdout}
\ImportTok{from}\NormalTok{ datetime }\ImportTok{import}\NormalTok{ datetime}
\ImportTok{from}\NormalTok{ typing }\ImportTok{import}\NormalTok{ Callable}
\ImportTok{from}\NormalTok{ math }\ImportTok{import}\NormalTok{ exp, cos, sin}

\NormalTok{logging.basicConfig(}
\NormalTok{    level}\OperatorTok{=}\NormalTok{logging.INFO,}
    \BuiltInTok{format}\OperatorTok{=}\StringTok{"[}\SpecialCharTok{\%(asctime)s}\StringTok{][}\SpecialCharTok{\%(levelname)s}\StringTok{] }\SpecialCharTok{\%(message)s}\StringTok{"}\NormalTok{,}
\NormalTok{    stream}\OperatorTok{=}\NormalTok{stdout,}
\NormalTok{    datefmt}\OperatorTok{=}\StringTok{"\%m{-}}\SpecialCharTok{\%d}\StringTok{ \%H:\%M:\%S"}\NormalTok{,}
\NormalTok{)}

\NormalTok{logging.info(datetime.now())}

\KeywordTok{def}\NormalTok{ ODE\_euler(}
    \OperatorTok{*}\NormalTok{,}
\NormalTok{    a: }\BuiltInTok{float}\NormalTok{,}
\NormalTok{    b: }\BuiltInTok{float}\NormalTok{,}
\NormalTok{    f: Callable[[}\BuiltInTok{float}\NormalTok{, }\BuiltInTok{float}\NormalTok{], }\BuiltInTok{float}\NormalTok{],}
\NormalTok{    y\_t0: }\BuiltInTok{float}\NormalTok{,}
\NormalTok{    h: }\BuiltInTok{float}\NormalTok{,}
\NormalTok{) }\OperatorTok{{-}\textgreater{}} \BuiltInTok{tuple}\NormalTok{[}\BuiltInTok{list}\NormalTok{[}\BuiltInTok{float}\NormalTok{], }\BuiltInTok{list}\NormalTok{[}\BuiltInTok{float}\NormalTok{]]:}
\NormalTok{    N }\OperatorTok{=} \BuiltInTok{int}\NormalTok{((b }\OperatorTok{{-}}\NormalTok{ a) }\OperatorTok{/}\NormalTok{ h) }
\NormalTok{    t }\OperatorTok{=}\NormalTok{ a}
\NormalTok{    ts }\OperatorTok{=}\NormalTok{ [t]}
\NormalTok{    ys }\OperatorTok{=}\NormalTok{ [y\_t0]}

    \ControlFlowTok{for}\NormalTok{ \_ }\KeywordTok{in} \BuiltInTok{range}\NormalTok{(N):}
\NormalTok{        y }\OperatorTok{=}\NormalTok{ ys[}\OperatorTok{{-}}\DecValTok{1}\NormalTok{]}
\NormalTok{        y }\OperatorTok{+=}\NormalTok{ h }\OperatorTok{*}\NormalTok{ f(t, y)}
\NormalTok{        ys.append(y)}

\NormalTok{        t }\OperatorTok{+=}\NormalTok{ h}
\NormalTok{        ts.append(t)}
    \ControlFlowTok{return}\NormalTok{ ys, ts}
\end{Highlighting}
\end{Shaded}

\begin{verbatim}
[08-09 13:25:19][INFO] 2024-08-09 13:25:19.216937
\end{verbatim}

\subsubsection{\texorpdfstring{a)
\(y' = t e^{3t} - 2y , 0 \leq t \leq 1 ,  y(0) = 0\), con \$ h = 0.5
\$}{a) y\textquotesingle{} = t e\^{}\{3t\} - 2y , 0 \textbackslash leq t \textbackslash leq 1 ,  y(0) = 0, con \$ h = 0.5 \$}}\label{a-y-t-e3t---2y-0-leq-t-leq-1-y0-0-con-h-0.5}

\begin{Shaded}
\begin{Highlighting}[]
\KeywordTok{def}\NormalTok{ problem\_a(t, y):}
    \ControlFlowTok{return}\NormalTok{ t }\OperatorTok{*}\NormalTok{ exp(}\DecValTok{3} \OperatorTok{*}\NormalTok{ t) }\OperatorTok{{-}} \DecValTok{2} \OperatorTok{*}\NormalTok{ y}

\NormalTok{a }\OperatorTok{=} \FloatTok{0.0}  
\NormalTok{b }\OperatorTok{=} \FloatTok{1.0}  
\NormalTok{y\_t0 }\OperatorTok{=} \FloatTok{0.0}  
\NormalTok{h }\OperatorTok{=} \FloatTok{0.5}  
\NormalTok{ys\_a, ts\_a }\OperatorTok{=}\NormalTok{ ODE\_euler(a}\OperatorTok{=}\NormalTok{a, b}\OperatorTok{=}\NormalTok{b, f}\OperatorTok{=}\NormalTok{problem\_a, y\_t0}\OperatorTok{=}\NormalTok{y\_t0, h}\OperatorTok{=}\NormalTok{h)}
\NormalTok{logging.info(}\SpecialStringTok{f"Aproximaciones por el metodo de euler:"}\NormalTok{)}
\ControlFlowTok{for}\NormalTok{ t, y }\KeywordTok{in} \BuiltInTok{zip}\NormalTok{(ts\_a, ys\_a):}
\NormalTok{    logging.info(}\SpecialStringTok{f"t = }\SpecialCharTok{\{}\NormalTok{t}\SpecialCharTok{:.2f\}}\SpecialStringTok{, y = }\SpecialCharTok{\{}\NormalTok{y}\SpecialCharTok{:.4f\}}\SpecialStringTok{"}\NormalTok{)}
\end{Highlighting}
\end{Shaded}

\begin{verbatim}
[08-09 13:25:19][INFO] Aproximaciones por el metodo de euler:
[08-09 13:25:19][INFO] t = 0.00, y = 0.0000
[08-09 13:25:19][INFO] t = 0.50, y = 0.0000
[08-09 13:25:19][INFO] t = 1.00, y = 1.1204
\end{verbatim}

\subsubsection{\texorpdfstring{b)\(y' = 1+{(t-y)}^2, 2 \leq t \leq 3 ,  y(2) = 1\),
con \$ h = 0.5
\$}{b)y\textquotesingle{} = 1+\{(t-y)\}\^{}2, 2 \textbackslash leq t \textbackslash leq 3 ,  y(2) = 1, con \$ h = 0.5 \$}}\label{by-1t-y2-2-leq-t-leq-3-y2-1-con-h-0.5}

\begin{Shaded}
\begin{Highlighting}[]
\KeywordTok{def}\NormalTok{ problem\_b(t, y):}
    \ControlFlowTok{return} \DecValTok{1} \OperatorTok{+}\NormalTok{ (t }\OperatorTok{{-}}\NormalTok{ y)}\OperatorTok{**}\DecValTok{2}
\NormalTok{a }\OperatorTok{=} \FloatTok{2.0}  
\NormalTok{b }\OperatorTok{=} \FloatTok{3.0}  
\NormalTok{y\_t0 }\OperatorTok{=} \FloatTok{1.0}  
\NormalTok{h }\OperatorTok{=} \FloatTok{0.5}  
\NormalTok{ys\_b, ts\_b }\OperatorTok{=}\NormalTok{ ODE\_euler(a}\OperatorTok{=}\NormalTok{a, b}\OperatorTok{=}\NormalTok{b, f}\OperatorTok{=}\NormalTok{problem\_b, y\_t0}\OperatorTok{=}\NormalTok{y\_t0, h}\OperatorTok{=}\NormalTok{h)}

\NormalTok{logging.info(}\SpecialStringTok{f"Aproximaciones por el metodo de Euler:"}\NormalTok{)}
\ControlFlowTok{for}\NormalTok{ t, y }\KeywordTok{in} \BuiltInTok{zip}\NormalTok{(ts\_b, ys\_b):}
\NormalTok{    logging.info(}\SpecialStringTok{f"t = }\SpecialCharTok{\{}\NormalTok{t}\SpecialCharTok{:.2f\}}\SpecialStringTok{, y = }\SpecialCharTok{\{}\NormalTok{y}\SpecialCharTok{:.4f\}}\SpecialStringTok{"}\NormalTok{)}
\end{Highlighting}
\end{Shaded}

\begin{verbatim}
[08-09 13:25:20][INFO] Aproximaciones por el metodo de Euler:
[08-09 13:25:20][INFO] t = 2.00, y = 1.0000
[08-09 13:25:20][INFO] t = 2.50, y = 2.0000
[08-09 13:25:20][INFO] t = 3.00, y = 2.6250
\end{verbatim}

\subsubsection{\texorpdfstring{c)
\(y' = 1+y/t,1 \leq t \leq 2 ,  y(1) = 2\), con \$ h = 0.25
\$}{c) y\textquotesingle{} = 1+y/t,1 \textbackslash leq t \textbackslash leq 2 ,  y(1) = 2, con \$ h = 0.25 \$}}\label{c-y-1yt1-leq-t-leq-2-y1-2-con-h-0.25}

\begin{Shaded}
\begin{Highlighting}[]

\KeywordTok{def}\NormalTok{ problem\_c(t, y):}
    \ControlFlowTok{return} \DecValTok{1} \OperatorTok{+}\NormalTok{ y }\OperatorTok{/}\NormalTok{ t}

\NormalTok{a }\OperatorTok{=} \FloatTok{1.0}  
\NormalTok{b }\OperatorTok{=} \FloatTok{2.0}  
\NormalTok{y\_t0 }\OperatorTok{=} \FloatTok{2.0}  
\NormalTok{h }\OperatorTok{=} \FloatTok{0.25}  

\NormalTok{ys\_c, ts\_c }\OperatorTok{=}\NormalTok{ ODE\_euler(a}\OperatorTok{=}\NormalTok{a, b}\OperatorTok{=}\NormalTok{b, f}\OperatorTok{=}\NormalTok{problem\_c, y\_t0}\OperatorTok{=}\NormalTok{y\_t0, h}\OperatorTok{=}\NormalTok{h)}

\NormalTok{logging.info(}\SpecialStringTok{f"Aproximaciones por el metodo de Euler:"}\NormalTok{)}
\ControlFlowTok{for}\NormalTok{ t, y }\KeywordTok{in} \BuiltInTok{zip}\NormalTok{(ts\_c, ys\_c):}
\NormalTok{    logging.info(}\SpecialStringTok{f"t = }\SpecialCharTok{\{}\NormalTok{t}\SpecialCharTok{:.2f\}}\SpecialStringTok{, y = }\SpecialCharTok{\{}\NormalTok{y}\SpecialCharTok{:.4f\}}\SpecialStringTok{"}\NormalTok{)}
\end{Highlighting}
\end{Shaded}

\begin{verbatim}
[08-09 13:25:21][INFO] Aproximaciones por el metodo de Euler:
[08-09 13:25:21][INFO] t = 1.00, y = 2.0000
[08-09 13:25:21][INFO] t = 1.25, y = 2.7500
[08-09 13:25:21][INFO] t = 1.50, y = 3.5500
[08-09 13:25:21][INFO] t = 1.75, y = 4.3917
[08-09 13:25:21][INFO] t = 2.00, y = 5.2690
\end{verbatim}

\subsubsection{\texorpdfstring{d)
\(y' = cos(2t)+sen(3t) , 0 \leq t \leq 1 ,  y(0) = 1\), con \$ h = 0.25
\$}{d) y\textquotesingle{} = cos(2t)+sen(3t) , 0 \textbackslash leq t \textbackslash leq 1 ,  y(0) = 1, con \$ h = 0.25 \$}}\label{d-y-cos2tsen3t-0-leq-t-leq-1-y0-1-con-h-0.25}

\begin{Shaded}
\begin{Highlighting}[]
\KeywordTok{def}\NormalTok{ problem\_d(t, y):}
    \ControlFlowTok{return}\NormalTok{ cos(}\DecValTok{2} \OperatorTok{*}\NormalTok{ t) }\OperatorTok{+}\NormalTok{ sin(}\DecValTok{3} \OperatorTok{*}\NormalTok{ t)}

\CommentTok{\# Parámetros}
\NormalTok{a }\OperatorTok{=} \FloatTok{0.0}  
\NormalTok{b }\OperatorTok{=} \FloatTok{1.0}  
\NormalTok{y\_t0 }\OperatorTok{=} \FloatTok{1.0}  
\NormalTok{h }\OperatorTok{=} \FloatTok{0.25}  
\NormalTok{ys\_d, ts\_d }\OperatorTok{=}\NormalTok{ ODE\_euler(a}\OperatorTok{=}\NormalTok{a, b}\OperatorTok{=}\NormalTok{b, f}\OperatorTok{=}\NormalTok{problem\_d, y\_t0}\OperatorTok{=}\NormalTok{y\_t0, h}\OperatorTok{=}\NormalTok{h)}

\NormalTok{logging.info(}\SpecialStringTok{f"Aproximaciones por el metodo de Euler:"}\NormalTok{)}
\ControlFlowTok{for}\NormalTok{ t, y }\KeywordTok{in} \BuiltInTok{zip}\NormalTok{(ts\_d, ys\_d):}
\NormalTok{    logging.info(}\SpecialStringTok{f"t = }\SpecialCharTok{\{}\NormalTok{t}\SpecialCharTok{:.2f\}}\SpecialStringTok{, y = }\SpecialCharTok{\{}\NormalTok{y}\SpecialCharTok{:.4f\}}\SpecialStringTok{"}\NormalTok{)}
\end{Highlighting}
\end{Shaded}

\begin{verbatim}
[08-09 13:25:24][INFO] Aproximaciones por el metodo de Euler:
[08-09 13:25:24][INFO] t = 0.00, y = 1.0000
[08-09 13:25:24][INFO] t = 0.25, y = 1.2500
[08-09 13:25:24][INFO] t = 0.50, y = 1.6398
[08-09 13:25:24][INFO] t = 0.75, y = 2.0243
[08-09 13:25:24][INFO] t = 1.00, y = 2.2365
\end{verbatim}

\subsection{2. Las soluciones reales para los problemas de valor inicial
en el ejercicio 1 se proporcionan aquí. Compare el error real en cada
paso.}\label{las-soluciones-reales-para-los-problemas-de-valor-inicial-en-el-ejercicio-1-se-proporcionan-aquuxed.-compare-el-error-real-en-cada-paso.}

\subsubsection{\texorpdfstring{a)
\(y(t)=1/5te^{3t}-1/25e^{3t}+1/25e^{-2t}\)}{a) y(t)=1/5te\^{}\{3t\}-1/25e\^{}\{3t\}+1/25e\^{}\{-2t\}}}\label{a-yt15te3t-125e3t125e-2t}

\begin{Shaded}
\begin{Highlighting}[]
\KeywordTok{def}\NormalTok{ problem\_a(t, y):}
    \ControlFlowTok{return}\NormalTok{ t }\OperatorTok{*}\NormalTok{ exp(}\DecValTok{3} \OperatorTok{*}\NormalTok{ t) }\OperatorTok{{-}} \DecValTok{2} \OperatorTok{*}\NormalTok{ y}

\KeywordTok{def}\NormalTok{ exact\_a(t):}
    \ControlFlowTok{return}\NormalTok{ (}\DecValTok{1} \OperatorTok{/} \DecValTok{5}\NormalTok{) }\OperatorTok{*}\NormalTok{ t }\OperatorTok{*}\NormalTok{ exp(}\DecValTok{3} \OperatorTok{*}\NormalTok{ t) }\OperatorTok{{-}}\NormalTok{ (}\DecValTok{1} \OperatorTok{/} \DecValTok{25}\NormalTok{) }\OperatorTok{*}\NormalTok{ exp(}\DecValTok{3} \OperatorTok{*}\NormalTok{ t) }\OperatorTok{+}\NormalTok{ (}\DecValTok{1} \OperatorTok{/} \DecValTok{25}\NormalTok{) }\OperatorTok{*}\NormalTok{ exp(}\OperatorTok{{-}}\DecValTok{2} \OperatorTok{*}\NormalTok{ t)}
\NormalTok{a }\OperatorTok{=} \FloatTok{0.0}
\NormalTok{b }\OperatorTok{=} \FloatTok{1.0}
\NormalTok{y\_t0 }\OperatorTok{=} \FloatTok{0.0}
\NormalTok{h }\OperatorTok{=} \FloatTok{0.5}
\NormalTok{ys\_a, ts\_a }\OperatorTok{=}\NormalTok{ ODE\_euler(a}\OperatorTok{=}\NormalTok{a, b}\OperatorTok{=}\NormalTok{b, f}\OperatorTok{=}\NormalTok{problem\_a, y\_t0}\OperatorTok{=}\NormalTok{y\_t0, h}\OperatorTok{=}\NormalTok{h)}
\NormalTok{logging.info(}\SpecialStringTok{f"Resultados del problema (a) usando el método de Euler:"}\NormalTok{)}
\ControlFlowTok{for}\NormalTok{ t, y }\KeywordTok{in} \BuiltInTok{zip}\NormalTok{(ts\_a, ys\_a):}
\NormalTok{    exact\_y }\OperatorTok{=}\NormalTok{ exact\_a(t)}
\NormalTok{    error\_relativo }\OperatorTok{=} \BuiltInTok{abs}\NormalTok{((exact\_y }\OperatorTok{{-}}\NormalTok{ y) }\OperatorTok{/}\NormalTok{ exact\_y) }\ControlFlowTok{if}\NormalTok{ exact\_y }\OperatorTok{!=} \DecValTok{0} \ControlFlowTok{else} \BuiltInTok{float}\NormalTok{(}\StringTok{\textquotesingle{}inf\textquotesingle{}}\NormalTok{)}
\NormalTok{    logging.info(}\SpecialStringTok{f"t = }\SpecialCharTok{\{}\NormalTok{t}\SpecialCharTok{:.2f\}}\SpecialStringTok{, y = }\SpecialCharTok{\{}\NormalTok{y}\SpecialCharTok{:.4f\}}\SpecialStringTok{, exacta = }\SpecialCharTok{\{}\NormalTok{exact\_y}\SpecialCharTok{:.4f\}}\SpecialStringTok{, error relativo = }\SpecialCharTok{\{}\NormalTok{error\_relativo}\SpecialCharTok{:.4f\}}\SpecialStringTok{"}\NormalTok{)}
\end{Highlighting}
\end{Shaded}

\begin{verbatim}
TypeError: ODE_euler() got an unexpected keyword argument 'h'
---------------------------------------------------------------------------
TypeError                                 Traceback (most recent call last)
Cell In[423], line 10
      8 y_t0 = 0.0
      9 h = 0.5
---> 10 ys_a, ts_a = ODE_euler(a=a, b=b, f=problem_a, y_t0=y_t0, h=h)
     11 logging.info(f"Resultados del problema (a) usando el método de Euler:")
     12 for t, y in zip(ts_a, ys_a):

TypeError: ODE_euler() got an unexpected keyword argument 'h'
\end{verbatim}

\begin{Shaded}
\begin{Highlighting}[]
\NormalTok{f }\OperatorTok{=} \KeywordTok{lambda}\NormalTok{ t, y: t }\OperatorTok{*}\NormalTok{ exp(}\DecValTok{3} \OperatorTok{*}\NormalTok{ t) }\OperatorTok{{-}} \DecValTok{2} \OperatorTok{*}\NormalTok{ y}
\NormalTok{exact\_a }\OperatorTok{=} \KeywordTok{lambda}\NormalTok{ t: (}\DecValTok{1} \OperatorTok{/} \DecValTok{5}\NormalTok{) }\OperatorTok{*}\NormalTok{ t }\OperatorTok{*}\NormalTok{ exp(}\DecValTok{3} \OperatorTok{*}\NormalTok{ t) }\OperatorTok{{-}}\NormalTok{ (}\DecValTok{1} \OperatorTok{/} \DecValTok{25}\NormalTok{) }\OperatorTok{*}\NormalTok{ exp(}\DecValTok{3} \OperatorTok{*}\NormalTok{ t) }\OperatorTok{+}\NormalTok{ (}\DecValTok{1} \OperatorTok{/} \DecValTok{25}\NormalTok{) }\OperatorTok{*}\NormalTok{ exp(}\OperatorTok{{-}}\DecValTok{2} \OperatorTok{*}\NormalTok{ t)}
\NormalTok{a }\OperatorTok{=} \DecValTok{0}
\NormalTok{b }\OperatorTok{=} \DecValTok{1}
\NormalTok{y\_t0 }\OperatorTok{=}\DecValTok{0}
\NormalTok{h }\OperatorTok{=} \FloatTok{0.5}
\NormalTok{ys\_a, ts\_a }\OperatorTok{=}\NormalTok{ ODE\_euler(a}\OperatorTok{=}\NormalTok{a, b}\OperatorTok{=}\NormalTok{b, f}\OperatorTok{=}\NormalTok{f, y\_t0}\OperatorTok{=}\NormalTok{y\_t0, h}\OperatorTok{=}\NormalTok{h)}
\NormalTok{logging.info(}\SpecialStringTok{f"Solucion:"}\NormalTok{)}
\ControlFlowTok{for}\NormalTok{ t, y }\KeywordTok{in} \BuiltInTok{zip}\NormalTok{(ts\_a, ys\_a):}
\NormalTok{    exact\_y }\OperatorTok{=}\NormalTok{ exact\_a(t)}
\NormalTok{    error\_real }\OperatorTok{=} \BuiltInTok{abs}\NormalTok{(exact\_y }\OperatorTok{{-}}\NormalTok{ y)}
\NormalTok{    logging.info(}\SpecialStringTok{f"t = }\SpecialCharTok{\{}\NormalTok{t}\SpecialCharTok{:.2f\}}\SpecialStringTok{, y = }\SpecialCharTok{\{}\NormalTok{y}\SpecialCharTok{:.4f\}}\SpecialStringTok{, exacta = }\SpecialCharTok{\{}\NormalTok{exact\_y}\SpecialCharTok{:.4f\}}\SpecialStringTok{, error real = }\SpecialCharTok{\{}\NormalTok{error\_real}\SpecialCharTok{:.4f\}}\SpecialStringTok{"}\NormalTok{)}
\end{Highlighting}
\end{Shaded}

\begin{verbatim}
[08-09 14:10:57][INFO] Solucion:
[08-09 14:10:57][INFO] t = 0.00, y = 0.0000, exacta = 0.0000, error real = 0.0000
[08-09 14:10:57][INFO] t = 0.50, y = 0.0000, exacta = 0.2836, error real = 0.2836
[08-09 14:10:57][INFO] t = 1.00, y = 1.1204, exacta = 3.2191, error real = 2.0987
\end{verbatim}

\subsubsection{\texorpdfstring{b)
\(y(t)=t+\frac{1}{1-t}\)}{b) y(t)=t+\textbackslash frac\{1\}\{1-t\}}}\label{b-yttfrac11-t}

\begin{Shaded}
\begin{Highlighting}[]
\KeywordTok{def}\NormalTok{ problem\_b(t, y):}
    \ControlFlowTok{return} \DecValTok{1} \OperatorTok{+}\NormalTok{ (t }\OperatorTok{{-}}\NormalTok{ y)}\OperatorTok{**}\DecValTok{2}

\KeywordTok{def}\NormalTok{ exact\_b(t):}
    \ControlFlowTok{return}\NormalTok{ t }\OperatorTok{+} \DecValTok{1} \OperatorTok{/}\NormalTok{ (}\DecValTok{1} \OperatorTok{{-}}\NormalTok{ t)}
\NormalTok{a }\OperatorTok{=} \FloatTok{2.0}
\NormalTok{b }\OperatorTok{=} \FloatTok{3.0}
\NormalTok{y\_t0 }\OperatorTok{=} \FloatTok{1.0}
\NormalTok{h }\OperatorTok{=} \FloatTok{0.5}
\NormalTok{ys\_b, ts\_b }\OperatorTok{=}\NormalTok{ ODE\_euler(a}\OperatorTok{=}\NormalTok{a, b}\OperatorTok{=}\NormalTok{b, f}\OperatorTok{=}\NormalTok{problem\_b, y\_t0}\OperatorTok{=}\NormalTok{y\_t0, h}\OperatorTok{=}\NormalTok{h)}
\NormalTok{logging.info(}\SpecialStringTok{f"Resultados del problema (b) usando el método de Euler:"}\NormalTok{)}
\ControlFlowTok{for}\NormalTok{ t, y }\KeywordTok{in} \BuiltInTok{zip}\NormalTok{(ts\_b, ys\_b):}
\NormalTok{    exact\_y }\OperatorTok{=}\NormalTok{ exact\_b(t)}
\NormalTok{    error\_relativo }\OperatorTok{=} \BuiltInTok{abs}\NormalTok{((exact\_y }\OperatorTok{{-}}\NormalTok{ y) }\OperatorTok{/}\NormalTok{ exact\_y) }\ControlFlowTok{if}\NormalTok{ exact\_y }\OperatorTok{!=} \DecValTok{0} \ControlFlowTok{else} \BuiltInTok{float}\NormalTok{(}\StringTok{\textquotesingle{}inf\textquotesingle{}}\NormalTok{)}
\NormalTok{    logging.info(}\SpecialStringTok{f"t = }\SpecialCharTok{\{}\NormalTok{t}\SpecialCharTok{:.2f\}}\SpecialStringTok{, y = }\SpecialCharTok{\{}\NormalTok{y}\SpecialCharTok{:.4f\}}\SpecialStringTok{, exacta = }\SpecialCharTok{\{}\NormalTok{exact\_y}\SpecialCharTok{:.4f\}}\SpecialStringTok{, error relativo = }\SpecialCharTok{\{}\NormalTok{error\_relativo}\SpecialCharTok{:.4f\}}\SpecialStringTok{"}\NormalTok{)}
\end{Highlighting}
\end{Shaded}

\begin{verbatim}
[08-09 13:38:32][INFO] Resultados del problema (b) usando el método de Euler:
[08-09 13:38:32][INFO] t = 2.00, y = 1.0000, exacta = 1.0000, error relativo = 0.0000
[08-09 13:38:32][INFO] t = 2.50, y = 2.0000, exacta = 1.8333, error relativo = 0.0909
[08-09 13:38:32][INFO] t = 3.00, y = 2.6250, exacta = 2.5000, error relativo = 0.0500
\end{verbatim}

\subsubsection{c) y(t)=tln(t)+2t}\label{c-yttlnt2t}

\begin{Shaded}
\begin{Highlighting}[]
\KeywordTok{def}\NormalTok{ problem\_c(t, y):}
    \ControlFlowTok{return} \DecValTok{1} \OperatorTok{+}\NormalTok{ y }\OperatorTok{/}\NormalTok{ t}
\KeywordTok{def}\NormalTok{ exact\_c(t):}
    \ControlFlowTok{return}\NormalTok{ t }\OperatorTok{*}\NormalTok{ log(t) }\OperatorTok{+} \DecValTok{2} \OperatorTok{*}\NormalTok{ t}
\NormalTok{a }\OperatorTok{=} \FloatTok{1.0}
\NormalTok{b }\OperatorTok{=} \FloatTok{2.0}
\NormalTok{y\_t0 }\OperatorTok{=} \FloatTok{2.0}
\NormalTok{h }\OperatorTok{=} \FloatTok{0.25}
\NormalTok{ys\_c, ts\_c }\OperatorTok{=}\NormalTok{ ODE\_euler(a}\OperatorTok{=}\NormalTok{a, b}\OperatorTok{=}\NormalTok{b, f}\OperatorTok{=}\NormalTok{problem\_c, y\_t0}\OperatorTok{=}\NormalTok{y\_t0, h}\OperatorTok{=}\NormalTok{h)}
\NormalTok{logging.info(}\SpecialStringTok{f"Resultados del problema (c) usando el método de Euler:"}\NormalTok{)}
\ControlFlowTok{for}\NormalTok{ t, y }\KeywordTok{in} \BuiltInTok{zip}\NormalTok{(ts\_c, ys\_c):}
\NormalTok{    exact\_y }\OperatorTok{=}\NormalTok{ exact\_c(t)}
\NormalTok{    error\_relativo }\OperatorTok{=} \BuiltInTok{abs}\NormalTok{((exact\_y }\OperatorTok{{-}}\NormalTok{ y) }\OperatorTok{/}\NormalTok{ exact\_y) }\ControlFlowTok{if}\NormalTok{ exact\_y }\OperatorTok{!=} \DecValTok{0} \ControlFlowTok{else} \BuiltInTok{float}\NormalTok{(}\StringTok{\textquotesingle{}inf\textquotesingle{}}\NormalTok{)}
\NormalTok{    logging.info(}\SpecialStringTok{f"t = }\SpecialCharTok{\{}\NormalTok{t}\SpecialCharTok{:.2f\}}\SpecialStringTok{, y = }\SpecialCharTok{\{}\NormalTok{y}\SpecialCharTok{:.4f\}}\SpecialStringTok{, exacta = }\SpecialCharTok{\{}\NormalTok{exact\_y}\SpecialCharTok{:.4f\}}\SpecialStringTok{, error relativo = }\SpecialCharTok{\{}\NormalTok{error\_relativo}\SpecialCharTok{:.4f\}}\SpecialStringTok{"}\NormalTok{)}
\end{Highlighting}
\end{Shaded}

\begin{verbatim}
[08-09 13:38:36][INFO] Resultados del problema (c) usando el método de Euler:
[08-09 13:38:36][INFO] t = 1.00, y = 2.0000, exacta = 2.0000, error relativo = 0.0000
[08-09 13:38:36][INFO] t = 1.25, y = 2.7500, exacta = 2.7789, error relativo = 0.0104
[08-09 13:38:36][INFO] t = 1.50, y = 3.5500, exacta = 3.6082, error relativo = 0.0161
[08-09 13:38:36][INFO] t = 1.75, y = 4.3917, exacta = 4.4793, error relativo = 0.0196
[08-09 13:38:36][INFO] t = 2.00, y = 5.2690, exacta = 5.3863, error relativo = 0.0218
\end{verbatim}

\subsubsection{\texorpdfstring{d)
\(y(t)=1/2sen(2t)-1/3cos(3t)+4/3\)}{d) y(t)=1/2sen(2t)-1/3cos(3t)+4/3}}\label{d-yt12sen2t-13cos3t43}

\begin{Shaded}
\begin{Highlighting}[]
\KeywordTok{def}\NormalTok{ problem\_d(t, y):}
    \ControlFlowTok{return}\NormalTok{ cos(}\DecValTok{2} \OperatorTok{*}\NormalTok{ t) }\OperatorTok{+}\NormalTok{ sin(}\DecValTok{3} \OperatorTok{*}\NormalTok{ t)}
\KeywordTok{def}\NormalTok{ exact\_d(t):}
    \ControlFlowTok{return}\NormalTok{ (}\DecValTok{1}\OperatorTok{/}\DecValTok{5}\NormalTok{) }\OperatorTok{*}\NormalTok{ sin(}\DecValTok{2}\OperatorTok{*}\NormalTok{t) }\OperatorTok{{-}}\NormalTok{ (}\DecValTok{1}\OperatorTok{/}\DecValTok{10}\NormalTok{) }\OperatorTok{*}\NormalTok{ cos(}\DecValTok{2}\OperatorTok{*}\NormalTok{t) }\OperatorTok{{-}}\NormalTok{ (}\DecValTok{1}\OperatorTok{/}\DecValTok{9}\NormalTok{) }\OperatorTok{*}\NormalTok{ cos(}\DecValTok{3}\OperatorTok{*}\NormalTok{t) }\OperatorTok{+}\NormalTok{ (}\DecValTok{1}\OperatorTok{/}\DecValTok{27}\NormalTok{) }\OperatorTok{*}\NormalTok{ sin(}\DecValTok{3}\OperatorTok{*}\NormalTok{t) }\OperatorTok{+} \DecValTok{1}
\NormalTok{a }\OperatorTok{=} \FloatTok{0.0}
\NormalTok{b }\OperatorTok{=} \FloatTok{1.0}
\NormalTok{y\_t0 }\OperatorTok{=} \FloatTok{1.0}
\NormalTok{h }\OperatorTok{=} \FloatTok{0.25}
\NormalTok{ys\_d, ts\_d }\OperatorTok{=}\NormalTok{ ODE\_euler(a}\OperatorTok{=}\NormalTok{a, b}\OperatorTok{=}\NormalTok{b, f}\OperatorTok{=}\NormalTok{problem\_d, y\_t0}\OperatorTok{=}\NormalTok{y\_t0, h}\OperatorTok{=}\NormalTok{h)}
\NormalTok{logging.info(}\SpecialStringTok{f"Resultados del problema (d) usando el método de Euler:"}\NormalTok{)}
\ControlFlowTok{for}\NormalTok{ t, y }\KeywordTok{in} \BuiltInTok{zip}\NormalTok{(ts\_d, ys\_d):}
\NormalTok{    exact\_y }\OperatorTok{=}\NormalTok{ exact\_d(t)}
\NormalTok{    error\_relativo }\OperatorTok{=} \BuiltInTok{abs}\NormalTok{((exact\_y }\OperatorTok{{-}}\NormalTok{ y) }\OperatorTok{/}\NormalTok{ exact\_y) }\ControlFlowTok{if}\NormalTok{ exact\_y }\OperatorTok{!=} \DecValTok{0} \ControlFlowTok{else} \BuiltInTok{float}\NormalTok{(}\StringTok{\textquotesingle{}inf\textquotesingle{}}\NormalTok{)}
\NormalTok{    logging.info(}\SpecialStringTok{f"t = }\SpecialCharTok{\{}\NormalTok{t}\SpecialCharTok{:.2f\}}\SpecialStringTok{, y = }\SpecialCharTok{\{}\NormalTok{y}\SpecialCharTok{:.4f\}}\SpecialStringTok{, exacta = }\SpecialCharTok{\{}\NormalTok{exact\_y}\SpecialCharTok{:.4f\}}\SpecialStringTok{, error relativo = }\SpecialCharTok{\{}\NormalTok{error\_relativo}\SpecialCharTok{:.4f\}}\SpecialStringTok{"}\NormalTok{)}
\end{Highlighting}
\end{Shaded}

\begin{verbatim}
[08-09 13:38:40][INFO] Resultados del problema (d) usando el método de Euler:
[08-09 13:38:40][INFO] t = 0.00, y = 1.0000, exacta = 0.7889, error relativo = 0.2676
[08-09 13:38:40][INFO] t = 0.25, y = 1.2500, exacta = 0.9521, error relativo = 0.3129
[08-09 13:38:40][INFO] t = 0.50, y = 1.6398, exacta = 1.1433, error relativo = 0.4342
[08-09 13:38:40][INFO] t = 0.75, y = 2.0243, exacta = 1.2910, error relativo = 0.5679
[08-09 13:38:40][INFO] t = 1.00, y = 2.2365, exacta = 1.3387, error relativo = 0.6706
\end{verbatim}

\subsection{3.Utilice el método de Euler para aproximar las soluciones
para cada uno de los siguientes problemas de valor
inicial.}\label{utilice-el-muxe9todo-de-euler-para-aproximar-las-soluciones-para-cada-uno-de-los-siguientes-problemas-de-valor-inicial.}

\subsubsection{\texorpdfstring{a)\(y' = y/t-(y/t)^2, 1 \leq t \leq 2 ,  y(1) = 1\),
con \$ h =
0.1\$}{a)y\textquotesingle{} = y/t-(y/t)\^{}2, 1 \textbackslash leq t \textbackslash leq 2 ,  y(1) = 1, con \$ h = 0.1\$}}\label{ay-yt-yt2-1-leq-t-leq-2-y1-1-con-h-0.1}

\begin{Shaded}
\begin{Highlighting}[]
\NormalTok{f }\OperatorTok{=} \KeywordTok{lambda}\NormalTok{ t, y: (y }\OperatorTok{/}\NormalTok{ t) }\OperatorTok{{-}}\NormalTok{ (y }\OperatorTok{/}\NormalTok{ t)}\OperatorTok{**}\DecValTok{2}
\NormalTok{a }\OperatorTok{=} \DecValTok{1} 
\NormalTok{b }\OperatorTok{=} \DecValTok{2}  
\NormalTok{y\_t0 }\OperatorTok{=} \DecValTok{1} 
\NormalTok{h }\OperatorTok{=} \FloatTok{0.1} 
\NormalTok{ys\_b, ts\_b }\OperatorTok{=}\NormalTok{ ODE\_euler(a}\OperatorTok{=}\NormalTok{a, b}\OperatorTok{=}\NormalTok{b, f}\OperatorTok{=}\NormalTok{f, y\_t0}\OperatorTok{=}\NormalTok{y\_t0, h}\OperatorTok{=}\NormalTok{h)}

\NormalTok{logging.info(}\SpecialStringTok{f"Aproximaciones por el metodo de Euler:"}\NormalTok{)}
\ControlFlowTok{for}\NormalTok{ t, y }\KeywordTok{in} \BuiltInTok{zip}\NormalTok{(ts\_b, ys\_b):}
\NormalTok{    logging.info(}\SpecialStringTok{f"t = }\SpecialCharTok{\{}\NormalTok{t}\SpecialCharTok{:.2f\}}\SpecialStringTok{, y = }\SpecialCharTok{\{}\NormalTok{y}\SpecialCharTok{:.4f\}}\SpecialStringTok{"}\NormalTok{)}
\end{Highlighting}
\end{Shaded}

\begin{verbatim}
[08-09 14:18:50][INFO] Aproximaciones por el metodo de Euler:
[08-09 14:18:50][INFO] t = 1.00, y = 1.0000
[08-09 14:18:50][INFO] t = 1.10, y = 1.0000
[08-09 14:18:50][INFO] t = 1.20, y = 1.0083
[08-09 14:18:50][INFO] t = 1.30, y = 1.0217
[08-09 14:18:50][INFO] t = 1.40, y = 1.0385
[08-09 14:18:50][INFO] t = 1.50, y = 1.0577
[08-09 14:18:50][INFO] t = 1.60, y = 1.0785
[08-09 14:18:50][INFO] t = 1.70, y = 1.1004
[08-09 14:18:50][INFO] t = 1.80, y = 1.1233
[08-09 14:18:50][INFO] t = 1.90, y = 1.1467
[08-09 14:18:50][INFO] t = 2.00, y = 1.1707
\end{verbatim}

\subsubsection{\texorpdfstring{b)\(y' = 1+y/t+(y/t)^2, 1 \leq t \leq 3 ,  y(1) = 0\),
con \$ h = 0.2
\$}{b)y\textquotesingle{} = 1+y/t+(y/t)\^{}2, 1 \textbackslash leq t \textbackslash leq 3 ,  y(1) = 0, con \$ h = 0.2 \$}}\label{by-1ytyt2-1-leq-t-leq-3-y1-0-con-h-0.2}

\begin{Shaded}
\begin{Highlighting}[]
\NormalTok{f }\OperatorTok{=} \KeywordTok{lambda}\NormalTok{ t, y: }\DecValTok{1}\OperatorTok{+}\NormalTok{y}\OperatorTok{/}\NormalTok{t}\OperatorTok{+}\NormalTok{(y}\OperatorTok{/}\NormalTok{t)}\OperatorTok{**}\DecValTok{2}
\NormalTok{a }\OperatorTok{=} \DecValTok{1} 
\NormalTok{b }\OperatorTok{=} \DecValTok{3} 
\NormalTok{y\_t0 }\OperatorTok{=} \DecValTok{0} 
\NormalTok{h }\OperatorTok{=} \FloatTok{0.2} 
\NormalTok{ys\_b, ts\_b }\OperatorTok{=}\NormalTok{ ODE\_euler(a}\OperatorTok{=}\NormalTok{a, b}\OperatorTok{=}\NormalTok{b, f}\OperatorTok{=}\NormalTok{f, y\_t0}\OperatorTok{=}\NormalTok{y\_t0, h}\OperatorTok{=}\NormalTok{h)}

\NormalTok{logging.info(}\SpecialStringTok{f"Aproximaciones por el metodo de Euler:"}\NormalTok{)}
\ControlFlowTok{for}\NormalTok{ t, y }\KeywordTok{in} \BuiltInTok{zip}\NormalTok{(ts\_b, ys\_b):}
\NormalTok{    logging.info(}\SpecialStringTok{f"t = }\SpecialCharTok{\{}\NormalTok{t}\SpecialCharTok{:.2f\}}\SpecialStringTok{, y = }\SpecialCharTok{\{}\NormalTok{y}\SpecialCharTok{:.4f\}}\SpecialStringTok{"}\NormalTok{)}
\end{Highlighting}
\end{Shaded}

\begin{verbatim}
[08-09 14:18:50][INFO] Aproximaciones por el metodo de Euler:
[08-09 14:18:50][INFO] t = 1.00, y = 0.0000
[08-09 14:18:50][INFO] t = 1.20, y = 0.2000
[08-09 14:18:50][INFO] t = 1.40, y = 0.4389
[08-09 14:18:50][INFO] t = 1.60, y = 0.7212
[08-09 14:18:50][INFO] t = 1.80, y = 1.0520
[08-09 14:18:50][INFO] t = 2.00, y = 1.4373
[08-09 14:18:50][INFO] t = 2.20, y = 1.8843
[08-09 14:18:50][INFO] t = 2.40, y = 2.4023
[08-09 14:18:50][INFO] t = 2.60, y = 3.0028
[08-09 14:18:50][INFO] t = 2.80, y = 3.7006
[08-09 14:18:50][INFO] t = 3.00, y = 4.5143
\end{verbatim}

\subsubsection{\texorpdfstring{c)\(y' = -(y+1)(y+3), 0 \leq t \leq 2 ,  y(0) = -2\),
con \$ h = 0.2
\$}{c)y\textquotesingle{} = -(y+1)(y+3), 0 \textbackslash leq t \textbackslash leq 2 ,  y(0) = -2, con \$ h = 0.2 \$}}\label{cy--y1y3-0-leq-t-leq-2-y0--2-con-h-0.2}

\begin{Shaded}
\begin{Highlighting}[]
\NormalTok{f }\OperatorTok{=} \KeywordTok{lambda}\NormalTok{ t, y: }\OperatorTok{{-}}\NormalTok{(y}\OperatorTok{+}\DecValTok{1}\NormalTok{)}\OperatorTok{*}\NormalTok{(y}\OperatorTok{+}\DecValTok{3}\NormalTok{)}
\NormalTok{a }\OperatorTok{=} \DecValTok{0} 
\NormalTok{b }\OperatorTok{=} \DecValTok{2}  
\NormalTok{y\_t0 }\OperatorTok{=} \OperatorTok{{-}}\DecValTok{2} 
\NormalTok{h }\OperatorTok{=} \FloatTok{0.2} 
\NormalTok{ys\_b, ts\_b }\OperatorTok{=}\NormalTok{ ODE\_euler(a}\OperatorTok{=}\NormalTok{a, b}\OperatorTok{=}\NormalTok{b, f}\OperatorTok{=}\NormalTok{f, y\_t0}\OperatorTok{=}\NormalTok{y\_t0, h}\OperatorTok{=}\NormalTok{h)}

\NormalTok{logging.info(}\SpecialStringTok{f"Aproximaciones por el metodo de Euler:"}\NormalTok{)}
\ControlFlowTok{for}\NormalTok{ t, y }\KeywordTok{in} \BuiltInTok{zip}\NormalTok{(ts\_b, ys\_b):}
\NormalTok{    logging.info(}\SpecialStringTok{f"t = }\SpecialCharTok{\{}\NormalTok{t}\SpecialCharTok{:.2f\}}\SpecialStringTok{, y = }\SpecialCharTok{\{}\NormalTok{y}\SpecialCharTok{:.4f\}}\SpecialStringTok{"}\NormalTok{)}
\end{Highlighting}
\end{Shaded}

\begin{verbatim}
[08-09 14:18:51][INFO] Aproximaciones por el metodo de Euler:
[08-09 14:18:51][INFO] t = 0.00, y = -2.0000
[08-09 14:18:51][INFO] t = 0.20, y = -1.8000
[08-09 14:18:51][INFO] t = 0.40, y = -1.6080
[08-09 14:18:51][INFO] t = 0.60, y = -1.4387
[08-09 14:18:51][INFO] t = 0.80, y = -1.3017
[08-09 14:18:51][INFO] t = 1.00, y = -1.1993
[08-09 14:18:51][INFO] t = 1.20, y = -1.1275
[08-09 14:18:51][INFO] t = 1.40, y = -1.0797
[08-09 14:18:51][INFO] t = 1.60, y = -1.0491
[08-09 14:18:51][INFO] t = 1.80, y = -1.0300
[08-09 14:18:51][INFO] t = 2.00, y = -1.0182
\end{verbatim}

\subsubsection{\texorpdfstring{d)\(y' = -5y+5t^2+2t, 0 \leq t \leq 1 ,  y(0) = 1/3\),
con \$ h = 0.1
\$}{d)y\textquotesingle{} = -5y+5t\^{}2+2t, 0 \textbackslash leq t \textbackslash leq 1 ,  y(0) = 1/3, con \$ h = 0.1 \$}}\label{dy--5y5t22t-0-leq-t-leq-1-y0-13-con-h-0.1}

\begin{Shaded}
\begin{Highlighting}[]
\NormalTok{f }\OperatorTok{=} \KeywordTok{lambda}\NormalTok{ t, y: }\OperatorTok{{-}}\DecValTok{5}\OperatorTok{*}\NormalTok{y}\OperatorTok{+}\DecValTok{5}\OperatorTok{*}\NormalTok{t}\OperatorTok{**}\DecValTok{2}\OperatorTok{+}\DecValTok{2}\OperatorTok{*}\NormalTok{t}
\NormalTok{a }\OperatorTok{=} \DecValTok{0} 
\NormalTok{b }\OperatorTok{=} \DecValTok{1}  
\NormalTok{y\_t0 }\OperatorTok{=} \DecValTok{1}\OperatorTok{/}\DecValTok{3} 
\NormalTok{h }\OperatorTok{=} \FloatTok{0.1} 
\NormalTok{ys\_b, ts\_b }\OperatorTok{=}\NormalTok{ ODE\_euler(a}\OperatorTok{=}\NormalTok{a, b}\OperatorTok{=}\NormalTok{b, f}\OperatorTok{=}\NormalTok{f, y\_t0}\OperatorTok{=}\NormalTok{y\_t0, h}\OperatorTok{=}\NormalTok{h)}

\NormalTok{logging.info(}\SpecialStringTok{f"Aproximaciones por el metodo de Euler:"}\NormalTok{)}
\ControlFlowTok{for}\NormalTok{ t, y }\KeywordTok{in} \BuiltInTok{zip}\NormalTok{(ts\_b, ys\_b):}
\NormalTok{    logging.info(}\SpecialStringTok{f"t = }\SpecialCharTok{\{}\NormalTok{t}\SpecialCharTok{:.2f\}}\SpecialStringTok{, y = }\SpecialCharTok{\{}\NormalTok{y}\SpecialCharTok{:.4f\}}\SpecialStringTok{"}\NormalTok{)}
\end{Highlighting}
\end{Shaded}

\begin{verbatim}
[08-09 14:18:51][INFO] Aproximaciones por el metodo de Euler:
[08-09 14:18:51][INFO] t = 0.00, y = 0.3333
[08-09 14:18:51][INFO] t = 0.10, y = 0.1667
[08-09 14:18:51][INFO] t = 0.20, y = 0.1083
[08-09 14:18:51][INFO] t = 0.30, y = 0.1142
[08-09 14:18:51][INFO] t = 0.40, y = 0.1621
[08-09 14:18:51][INFO] t = 0.50, y = 0.2410
[08-09 14:18:51][INFO] t = 0.60, y = 0.3455
[08-09 14:18:51][INFO] t = 0.70, y = 0.4728
[08-09 14:18:51][INFO] t = 0.80, y = 0.6214
[08-09 14:18:51][INFO] t = 0.90, y = 0.7907
[08-09 14:18:51][INFO] t = 1.00, y = 0.9803
\end{verbatim}

\subsection{4. Aquí se dan las soluciones reales para los problemas de
valor inicial en el ejercicio 3. Calcule el error real en las
aproximaciones del ejercicio
3.}\label{aquuxed-se-dan-las-soluciones-reales-para-los-problemas-de-valor-inicial-en-el-ejercicio-3.-calcule-el-error-real-en-las-aproximaciones-del-ejercicio-3.}

\begin{Shaded}
\begin{Highlighting}[]
\ImportTok{from}\NormalTok{ math }\ImportTok{import}\NormalTok{ log, tan, exp}
\end{Highlighting}
\end{Shaded}

\subsubsection{\texorpdfstring{a)
\(y(t)=\frac{t}{1+lnt}\)}{a) y(t)=\textbackslash frac\{t\}\{1+lnt\}}}\label{a-ytfract1lnt}

\begin{Shaded}
\begin{Highlighting}[]
\NormalTok{f }\OperatorTok{=} \KeywordTok{lambda}\NormalTok{ t, y: (y }\OperatorTok{/}\NormalTok{ t) }\OperatorTok{{-}}\NormalTok{ (y }\OperatorTok{/}\NormalTok{ t)}\OperatorTok{**}\DecValTok{2}
\NormalTok{exact\_a }\OperatorTok{=} \KeywordTok{lambda}\NormalTok{ t: t }\OperatorTok{/}\NormalTok{ (}\DecValTok{1} \OperatorTok{+}\NormalTok{ log(t))}
\NormalTok{a }\OperatorTok{=} \FloatTok{1.0}
\NormalTok{b }\OperatorTok{=} \FloatTok{2.0}
\NormalTok{y\_t0 }\OperatorTok{=} \FloatTok{1.0}
\NormalTok{h }\OperatorTok{=} \FloatTok{0.1}
\NormalTok{ys\_a, ts\_a }\OperatorTok{=}\NormalTok{ ODE\_euler(a}\OperatorTok{=}\NormalTok{a, b}\OperatorTok{=}\NormalTok{b, f}\OperatorTok{=}\NormalTok{f, y\_t0}\OperatorTok{=}\NormalTok{y\_t0, h}\OperatorTok{=}\NormalTok{h)}
\NormalTok{logging.info(}\SpecialStringTok{f"Solucion:"}\NormalTok{)}
\ControlFlowTok{for}\NormalTok{ t, y }\KeywordTok{in} \BuiltInTok{zip}\NormalTok{(ts\_a, ys\_a):}
\NormalTok{    exact\_y }\OperatorTok{=}\NormalTok{ exact\_a(t)}
\NormalTok{    error\_real }\OperatorTok{=} \BuiltInTok{abs}\NormalTok{(exact\_y }\OperatorTok{{-}}\NormalTok{ y)}
\NormalTok{    logging.info(}\SpecialStringTok{f"t = }\SpecialCharTok{\{}\NormalTok{t}\SpecialCharTok{:.2f\}}\SpecialStringTok{, y = }\SpecialCharTok{\{}\NormalTok{y}\SpecialCharTok{:.4f\}}\SpecialStringTok{, exacta = }\SpecialCharTok{\{}\NormalTok{exact\_y}\SpecialCharTok{:.4f\}}\SpecialStringTok{, error real = }\SpecialCharTok{\{}\NormalTok{error\_real}\SpecialCharTok{:.4f\}}\SpecialStringTok{"}\NormalTok{)}
\end{Highlighting}
\end{Shaded}

\begin{verbatim}
[08-09 14:18:53][INFO] Solucion:
[08-09 14:18:53][INFO] t = 1.00, y = 1.0000, exacta = 1.0000, error real = 0.0000
[08-09 14:18:53][INFO] t = 1.10, y = 1.0000, exacta = 1.0043, error real = 0.0043
[08-09 14:18:53][INFO] t = 1.20, y = 1.0083, exacta = 1.0150, error real = 0.0067
[08-09 14:18:53][INFO] t = 1.30, y = 1.0217, exacta = 1.0298, error real = 0.0081
[08-09 14:18:53][INFO] t = 1.40, y = 1.0385, exacta = 1.0475, error real = 0.0090
[08-09 14:18:53][INFO] t = 1.50, y = 1.0577, exacta = 1.0673, error real = 0.0096
[08-09 14:18:53][INFO] t = 1.60, y = 1.0785, exacta = 1.0884, error real = 0.0100
[08-09 14:18:53][INFO] t = 1.70, y = 1.1004, exacta = 1.1107, error real = 0.0102
[08-09 14:18:53][INFO] t = 1.80, y = 1.1233, exacta = 1.1337, error real = 0.0104
[08-09 14:18:53][INFO] t = 1.90, y = 1.1467, exacta = 1.1572, error real = 0.0105
[08-09 14:18:53][INFO] t = 2.00, y = 1.1707, exacta = 1.1812, error real = 0.0106
\end{verbatim}

\subsubsection{\texorpdfstring{b)
\(y(t)=t *tan(ln t)\)}{b) y(t)=t *tan(ln t)}}\label{b-ytt-tanln-t}

\begin{Shaded}
\begin{Highlighting}[]
\NormalTok{f }\OperatorTok{=} \KeywordTok{lambda}\NormalTok{ t, y: }\DecValTok{1}\OperatorTok{+}\NormalTok{y}\OperatorTok{/}\NormalTok{t}\OperatorTok{+}\NormalTok{(y}\OperatorTok{/}\NormalTok{t)}\OperatorTok{**}\DecValTok{2}
\NormalTok{exact\_a }\OperatorTok{=} \KeywordTok{lambda}\NormalTok{ t: t }\OperatorTok{*}\NormalTok{ tan(log(t))}
\NormalTok{a }\OperatorTok{=} \FloatTok{1.0}
\NormalTok{b }\OperatorTok{=} \DecValTok{3}
\NormalTok{y\_t0 }\OperatorTok{=} \DecValTok{0}
\NormalTok{h }\OperatorTok{=} \FloatTok{0.2}
\NormalTok{ys\_a, ts\_a }\OperatorTok{=}\NormalTok{ ODE\_euler(a}\OperatorTok{=}\NormalTok{a, b}\OperatorTok{=}\NormalTok{b, f}\OperatorTok{=}\NormalTok{f, y\_t0}\OperatorTok{=}\NormalTok{y\_t0, h}\OperatorTok{=}\NormalTok{h)}
\NormalTok{logging.info(}\SpecialStringTok{f"Solucion:"}\NormalTok{)}
\ControlFlowTok{for}\NormalTok{ t, y }\KeywordTok{in} \BuiltInTok{zip}\NormalTok{(ts\_a, ys\_a):}
\NormalTok{    exact\_y }\OperatorTok{=}\NormalTok{ exact\_a(t)}
\NormalTok{    error\_real }\OperatorTok{=} \BuiltInTok{abs}\NormalTok{(exact\_y }\OperatorTok{{-}}\NormalTok{ y)}
\NormalTok{    logging.info(}\SpecialStringTok{f"t = }\SpecialCharTok{\{}\NormalTok{t}\SpecialCharTok{:.2f\}}\SpecialStringTok{, y = }\SpecialCharTok{\{}\NormalTok{y}\SpecialCharTok{:.4f\}}\SpecialStringTok{, exacta = }\SpecialCharTok{\{}\NormalTok{exact\_y}\SpecialCharTok{:.4f\}}\SpecialStringTok{, error real = }\SpecialCharTok{\{}\NormalTok{error\_real}\SpecialCharTok{:.4f\}}\SpecialStringTok{"}\NormalTok{)}
\end{Highlighting}
\end{Shaded}

\begin{verbatim}
[08-09 14:18:55][INFO] Solucion:
[08-09 14:18:55][INFO] t = 1.00, y = 0.0000, exacta = 0.0000, error real = 0.0000
[08-09 14:18:55][INFO] t = 1.20, y = 0.2000, exacta = 0.2212, error real = 0.0212
[08-09 14:18:55][INFO] t = 1.40, y = 0.4389, exacta = 0.4897, error real = 0.0508
[08-09 14:18:55][INFO] t = 1.60, y = 0.7212, exacta = 0.8128, error real = 0.0915
[08-09 14:18:55][INFO] t = 1.80, y = 1.0520, exacta = 1.1994, error real = 0.1474
[08-09 14:18:55][INFO] t = 2.00, y = 1.4373, exacta = 1.6613, error real = 0.2240
[08-09 14:18:55][INFO] t = 2.20, y = 1.8843, exacta = 2.2135, error real = 0.3292
[08-09 14:18:55][INFO] t = 2.40, y = 2.4023, exacta = 2.8766, error real = 0.4743
[08-09 14:18:55][INFO] t = 2.60, y = 3.0028, exacta = 3.6785, error real = 0.6756
[08-09 14:18:55][INFO] t = 2.80, y = 3.7006, exacta = 4.6587, error real = 0.9581
[08-09 14:18:55][INFO] t = 3.00, y = 4.5143, exacta = 5.8741, error real = 1.3598
\end{verbatim}

\subsubsection{\texorpdfstring{c)
\(y(t)=-3+\frac{2}{1+e^{-2t}}\)}{c) y(t)=-3+\textbackslash frac\{2\}\{1+e\^{}\{-2t\}\}}}\label{c-yt-3frac21e-2t}

\begin{Shaded}
\begin{Highlighting}[]
\NormalTok{f }\OperatorTok{=} \KeywordTok{lambda}\NormalTok{ t, y: }\OperatorTok{{-}}\NormalTok{(y}\OperatorTok{+}\DecValTok{1}\NormalTok{)}\OperatorTok{*}\NormalTok{(y}\OperatorTok{+}\DecValTok{3}\NormalTok{)}
\NormalTok{exact\_a }\OperatorTok{=} \KeywordTok{lambda}\NormalTok{ t: }\OperatorTok{{-}}\DecValTok{3}\OperatorTok{+}\DecValTok{2}\OperatorTok{/}\NormalTok{(}\DecValTok{1}\OperatorTok{+}\NormalTok{exp(}\OperatorTok{{-}}\DecValTok{2}\OperatorTok{*}\NormalTok{t))}
\NormalTok{a }\OperatorTok{=} \DecValTok{0}
\NormalTok{b }\OperatorTok{=} \DecValTok{2}
\NormalTok{y\_t0 }\OperatorTok{=} \OperatorTok{{-}}\DecValTok{2}
\NormalTok{h }\OperatorTok{=} \FloatTok{0.2}
\NormalTok{ys\_a, ts\_a }\OperatorTok{=}\NormalTok{ ODE\_euler(a}\OperatorTok{=}\NormalTok{a, b}\OperatorTok{=}\NormalTok{b, f}\OperatorTok{=}\NormalTok{f, y\_t0}\OperatorTok{=}\NormalTok{y\_t0, h}\OperatorTok{=}\NormalTok{h)}
\NormalTok{logging.info(}\SpecialStringTok{f"Solucion:"}\NormalTok{)}
\ControlFlowTok{for}\NormalTok{ t, y }\KeywordTok{in} \BuiltInTok{zip}\NormalTok{(ts\_a, ys\_a):}
\NormalTok{    exact\_y }\OperatorTok{=}\NormalTok{ exact\_a(t)}
\NormalTok{    error\_real }\OperatorTok{=} \BuiltInTok{abs}\NormalTok{(exact\_y }\OperatorTok{{-}}\NormalTok{ y)}
\NormalTok{    logging.info(}\SpecialStringTok{f"t = }\SpecialCharTok{\{}\NormalTok{t}\SpecialCharTok{:.2f\}}\SpecialStringTok{, y = }\SpecialCharTok{\{}\NormalTok{y}\SpecialCharTok{:.4f\}}\SpecialStringTok{, exacta = }\SpecialCharTok{\{}\NormalTok{exact\_y}\SpecialCharTok{:.4f\}}\SpecialStringTok{, error real = }\SpecialCharTok{\{}\NormalTok{error\_real}\SpecialCharTok{:.4f\}}\SpecialStringTok{"}\NormalTok{)}
\end{Highlighting}
\end{Shaded}

\begin{verbatim}
[08-09 14:18:55][INFO] Solucion:
[08-09 14:18:55][INFO] t = 0.00, y = -2.0000, exacta = -2.0000, error real = 0.0000
[08-09 14:18:55][INFO] t = 0.20, y = -1.8000, exacta = -1.8026, error real = 0.0026
[08-09 14:18:55][INFO] t = 0.40, y = -1.6080, exacta = -1.6201, error real = 0.0121
[08-09 14:18:55][INFO] t = 0.60, y = -1.4387, exacta = -1.4630, error real = 0.0242
[08-09 14:18:55][INFO] t = 0.80, y = -1.3017, exacta = -1.3360, error real = 0.0342
[08-09 14:18:55][INFO] t = 1.00, y = -1.1993, exacta = -1.2384, error real = 0.0392
[08-09 14:18:55][INFO] t = 1.20, y = -1.1275, exacta = -1.1663, error real = 0.0389
[08-09 14:18:55][INFO] t = 1.40, y = -1.0797, exacta = -1.1146, error real = 0.0349
[08-09 14:18:55][INFO] t = 1.60, y = -1.0491, exacta = -1.0783, error real = 0.0292
[08-09 14:18:55][INFO] t = 1.80, y = -1.0300, exacta = -1.0532, error real = 0.0232
[08-09 14:18:55][INFO] t = 2.00, y = -1.0182, exacta = -1.0360, error real = 0.0178
\end{verbatim}

\subsubsection{\texorpdfstring{d)
\(y(t)=t^2+\frac{1}{3}e^{-5t}\)}{d) y(t)=t\^{}2+\textbackslash frac\{1\}\{3\}e\^{}\{-5t\}}}\label{d-ytt2frac13e-5t}

\begin{Shaded}
\begin{Highlighting}[]
\NormalTok{f }\OperatorTok{=} \KeywordTok{lambda}\NormalTok{ t, y: }\OperatorTok{{-}}\DecValTok{5}\OperatorTok{*}\NormalTok{y}\OperatorTok{+}\DecValTok{5}\OperatorTok{*}\NormalTok{t}\OperatorTok{**}\DecValTok{2}\OperatorTok{+}\DecValTok{2}\OperatorTok{*}\NormalTok{t}
\NormalTok{exact\_a }\OperatorTok{=} \KeywordTok{lambda}\NormalTok{ t: t}\OperatorTok{**}\DecValTok{2}\OperatorTok{+}\DecValTok{1}\OperatorTok{/}\DecValTok{3}\OperatorTok{*}\NormalTok{exp(}\OperatorTok{{-}}\DecValTok{5}\OperatorTok{*}\NormalTok{t)}
\NormalTok{a }\OperatorTok{=} \DecValTok{0}
\NormalTok{b }\OperatorTok{=} \DecValTok{1}
\NormalTok{y\_t0 }\OperatorTok{=}\DecValTok{1}\OperatorTok{/}\DecValTok{3}
\NormalTok{h }\OperatorTok{=} \FloatTok{0.1}
\NormalTok{ys\_a, ts\_a }\OperatorTok{=}\NormalTok{ ODE\_euler(a}\OperatorTok{=}\NormalTok{a, b}\OperatorTok{=}\NormalTok{b, f}\OperatorTok{=}\NormalTok{f, y\_t0}\OperatorTok{=}\NormalTok{y\_t0, h}\OperatorTok{=}\NormalTok{h)}
\NormalTok{logging.info(}\SpecialStringTok{f"Solucion:"}\NormalTok{)}
\ControlFlowTok{for}\NormalTok{ t, y }\KeywordTok{in} \BuiltInTok{zip}\NormalTok{(ts\_a, ys\_a):}
\NormalTok{    exact\_y }\OperatorTok{=}\NormalTok{ exact\_a(t)}
\NormalTok{    error\_real }\OperatorTok{=} \BuiltInTok{abs}\NormalTok{(exact\_y }\OperatorTok{{-}}\NormalTok{ y)}
\NormalTok{    logging.info(}\SpecialStringTok{f"t = }\SpecialCharTok{\{}\NormalTok{t}\SpecialCharTok{:.2f\}}\SpecialStringTok{, y = }\SpecialCharTok{\{}\NormalTok{y}\SpecialCharTok{:.4f\}}\SpecialStringTok{, exacta = }\SpecialCharTok{\{}\NormalTok{exact\_y}\SpecialCharTok{:.4f\}}\SpecialStringTok{, error real = }\SpecialCharTok{\{}\NormalTok{error\_real}\SpecialCharTok{:.4f\}}\SpecialStringTok{"}\NormalTok{)}
\end{Highlighting}
\end{Shaded}

\begin{verbatim}
[08-09 14:18:56][INFO] Solucion:
[08-09 14:18:56][INFO] t = 0.00, y = 0.3333, exacta = 0.3333, error real = 0.0000
[08-09 14:18:56][INFO] t = 0.10, y = 0.1667, exacta = 0.2122, error real = 0.0455
[08-09 14:18:56][INFO] t = 0.20, y = 0.1083, exacta = 0.1626, error real = 0.0543
[08-09 14:18:56][INFO] t = 0.30, y = 0.1142, exacta = 0.1644, error real = 0.0502
[08-09 14:18:56][INFO] t = 0.40, y = 0.1621, exacta = 0.2051, error real = 0.0430
[08-09 14:18:56][INFO] t = 0.50, y = 0.2410, exacta = 0.2774, error real = 0.0363
[08-09 14:18:56][INFO] t = 0.60, y = 0.3455, exacta = 0.3766, error real = 0.0311
[08-09 14:18:56][INFO] t = 0.70, y = 0.4728, exacta = 0.5001, error real = 0.0273
[08-09 14:18:56][INFO] t = 0.80, y = 0.6214, exacta = 0.6461, error real = 0.0247
[08-09 14:18:56][INFO] t = 0.90, y = 0.7907, exacta = 0.8137, error real = 0.0230
[08-09 14:18:56][INFO] t = 1.00, y = 0.9803, exacta = 1.0022, error real = 0.0219
\end{verbatim}

\subsection{5. Utilice los resultados del ejercicio 3 y la interpolación
lineal para aproximar los siguientes valores de 𝑦(𝑡). Compare las
aproximaciones asignadas para los valores reales obtenidos mediante las
funciones determinadas en el ejercicio
4.}\label{utilice-los-resultados-del-ejercicio-3-y-la-interpolaciuxf3n-lineal-para-aproximar-los-siguientes-valores-de-ux1d466ux1d461.-compare-las-aproximaciones-asignadas-para-los-valores-reales-obtenidos-mediante-las-funciones-determinadas-en-el-ejercicio-4.}

\begin{Shaded}
\begin{Highlighting}[]
\KeywordTok{def}\NormalTok{ interpolate\_linear(ts: List[}\BuiltInTok{float}\NormalTok{], ys: List[}\BuiltInTok{float}\NormalTok{], t: }\BuiltInTok{float}\NormalTok{) }\OperatorTok{{-}\textgreater{}} \BuiltInTok{float}\NormalTok{:}
    \ControlFlowTok{if}\NormalTok{ t }\OperatorTok{\textless{}}\NormalTok{ ts[}\DecValTok{0}\NormalTok{] }\KeywordTok{or}\NormalTok{ t }\OperatorTok{\textgreater{}}\NormalTok{ ts[}\OperatorTok{{-}}\DecValTok{1}\NormalTok{]:}
        \ControlFlowTok{raise} \PreprocessorTok{ValueError}\NormalTok{(}\StringTok{"El valor t está fuera del rango de los datos"}\NormalTok{)}
    
    \ControlFlowTok{for}\NormalTok{ i }\KeywordTok{in} \BuiltInTok{range}\NormalTok{(}\BuiltInTok{len}\NormalTok{(ts) }\OperatorTok{{-}} \DecValTok{1}\NormalTok{):}
        \ControlFlowTok{if}\NormalTok{ ts[i] }\OperatorTok{\textless{}=}\NormalTok{ t }\OperatorTok{\textless{}=}\NormalTok{ ts[i }\OperatorTok{+} \DecValTok{1}\NormalTok{]:}
\NormalTok{            t0, y0 }\OperatorTok{=}\NormalTok{ ts[i], ys[i]}
\NormalTok{            t1, y1 }\OperatorTok{=}\NormalTok{ ts[i }\OperatorTok{+} \DecValTok{1}\NormalTok{], ys[i }\OperatorTok{+} \DecValTok{1}\NormalTok{]}
            \ControlFlowTok{return}\NormalTok{ y0 }\OperatorTok{+}\NormalTok{ (t }\OperatorTok{{-}}\NormalTok{ t0) }\OperatorTok{/}\NormalTok{ (t1 }\OperatorTok{{-}}\NormalTok{ t0) }\OperatorTok{*}\NormalTok{ (y1 }\OperatorTok{{-}}\NormalTok{ y0)}
    \ControlFlowTok{if}\NormalTok{ t }\OperatorTok{\textless{}}\NormalTok{ ts[}\DecValTok{0}\NormalTok{]:}
        \ControlFlowTok{return}\NormalTok{ ys[}\DecValTok{0}\NormalTok{]}
    \ControlFlowTok{else}\NormalTok{:}
        \ControlFlowTok{return}\NormalTok{ ys[}\OperatorTok{{-}}\DecValTok{1}\NormalTok{]}
\KeywordTok{def}\NormalTok{ real\_a(t: }\BuiltInTok{float}\NormalTok{) }\OperatorTok{{-}\textgreater{}} \BuiltInTok{float}\NormalTok{:}
    \ControlFlowTok{return}\NormalTok{ t }\OperatorTok{/}\NormalTok{ (}\DecValTok{1} \OperatorTok{+}\NormalTok{ log(t))}

\KeywordTok{def}\NormalTok{ real\_b(t: }\BuiltInTok{float}\NormalTok{) }\OperatorTok{{-}\textgreater{}} \BuiltInTok{float}\NormalTok{:}
    \ControlFlowTok{return}\NormalTok{ t }\OperatorTok{*}\NormalTok{ tan(log(t))}

\KeywordTok{def}\NormalTok{ real\_c(t: }\BuiltInTok{float}\NormalTok{) }\OperatorTok{{-}\textgreater{}} \BuiltInTok{float}\NormalTok{:}
    \ControlFlowTok{return} \OperatorTok{{-}}\DecValTok{3} \OperatorTok{+} \DecValTok{2} \OperatorTok{/}\NormalTok{ (}\DecValTok{1} \OperatorTok{+}\NormalTok{ exp(}\OperatorTok{{-}}\DecValTok{2} \OperatorTok{*}\NormalTok{ t))}

\KeywordTok{def}\NormalTok{ real\_d(t: }\BuiltInTok{float}\NormalTok{) }\OperatorTok{{-}\textgreater{}} \BuiltInTok{float}\NormalTok{:}
    \ControlFlowTok{return}\NormalTok{ t }\OperatorTok{**} \DecValTok{2} \OperatorTok{+}\NormalTok{ (}\DecValTok{1} \OperatorTok{/} \DecValTok{3}\NormalTok{) }\OperatorTok{*}\NormalTok{ exp(}\OperatorTok{{-}}\DecValTok{5} \OperatorTok{*}\NormalTok{ t)}
\KeywordTok{def}\NormalTok{ interpolate\_and\_compare(ts: List[}\BuiltInTok{float}\NormalTok{], ys: List[}\BuiltInTok{float}\NormalTok{], t\_values: List[}\BuiltInTok{float}\NormalTok{], real\_func: Callable[[}\BuiltInTok{float}\NormalTok{], }\BuiltInTok{float}\NormalTok{]):}
\NormalTok{    results }\OperatorTok{=}\NormalTok{ []}
    \ControlFlowTok{for}\NormalTok{ t }\KeywordTok{in}\NormalTok{ t\_values:}
        \ControlFlowTok{try}\NormalTok{:}
\NormalTok{            interpolated\_y }\OperatorTok{=}\NormalTok{ interpolate\_linear(ts, ys, t)}
\NormalTok{            real\_y }\OperatorTok{=}\NormalTok{ real\_func(t)}
\NormalTok{            error }\OperatorTok{=} \BuiltInTok{abs}\NormalTok{(real\_y }\OperatorTok{{-}}\NormalTok{ interpolated\_y)}
\NormalTok{            results.append((t, interpolated\_y, real\_y, error))}
        \ControlFlowTok{except} \PreprocessorTok{ValueError} \ImportTok{as}\NormalTok{ e:}
\NormalTok{            logging.error(}\SpecialStringTok{f"Error en la interpolación: }\SpecialCharTok{\{}\NormalTok{e}\SpecialCharTok{\}}\SpecialStringTok{"}\NormalTok{)}
\NormalTok{            results.append((t, }\VariableTok{None}\NormalTok{, real\_func(t), }\VariableTok{None}\NormalTok{))}
    \ControlFlowTok{return}\NormalTok{ results}
\end{Highlighting}
\end{Shaded}

\subsubsection{\texorpdfstring{a) \(y(0.25)\) y
\(y(0.93)\)}{a) y(0.25) y y(0.93)}}\label{a-y0.25-y-y0.93}

\begin{Shaded}
\begin{Highlighting}[]
\NormalTok{a, b }\OperatorTok{=} \FloatTok{1.0}\NormalTok{, }\FloatTok{2.0}
\NormalTok{y\_t0 }\OperatorTok{=} \FloatTok{1.0}
\NormalTok{h }\OperatorTok{=} \FloatTok{0.1}
\NormalTok{f }\OperatorTok{=} \KeywordTok{lambda}\NormalTok{ t, y: (y }\OperatorTok{/}\NormalTok{ t) }\OperatorTok{{-}}\NormalTok{ (y }\OperatorTok{/}\NormalTok{ t) }\OperatorTok{**} \DecValTok{2}
\NormalTok{ys\_a, ts\_a }\OperatorTok{=}\NormalTok{ ODE\_euler(a}\OperatorTok{=}\NormalTok{a, b}\OperatorTok{=}\NormalTok{b, f}\OperatorTok{=}\NormalTok{f, y\_t0}\OperatorTok{=}\NormalTok{y\_t0, h}\OperatorTok{=}\NormalTok{h)}
\NormalTok{t\_values\_a }\OperatorTok{=}\NormalTok{ [}\FloatTok{1.25}\NormalTok{, }\FloatTok{1.93}\NormalTok{]}
\NormalTok{results\_a }\OperatorTok{=}\NormalTok{ interpolate\_and\_compare(ts\_a, ys\_a, t\_values\_a, real\_a)}
\NormalTok{logging.info(}\StringTok{"Resultados de interpolación y comparación con valores reales:"}\NormalTok{)}
\ControlFlowTok{for}\NormalTok{ t, interpolated, real, error }\KeywordTok{in}\NormalTok{ results\_a:}
    \ControlFlowTok{if}\NormalTok{ interpolated }\KeywordTok{is} \KeywordTok{not} \VariableTok{None}\NormalTok{:}
\NormalTok{        logging.info(}\SpecialStringTok{f"Problema (a): t = }\SpecialCharTok{\{}\NormalTok{t}\SpecialCharTok{:.2f\}}\SpecialStringTok{, Interpolado = }\SpecialCharTok{\{}\NormalTok{interpolated}\SpecialCharTok{:.6f\}}\SpecialStringTok{, Real = }\SpecialCharTok{\{}\NormalTok{real}\SpecialCharTok{:.6f\}}\SpecialStringTok{, Error = }\SpecialCharTok{\{}\NormalTok{error}\SpecialCharTok{:.6f\}}\SpecialStringTok{"}\NormalTok{)}
    \ControlFlowTok{else}\NormalTok{:}
\NormalTok{        logging.info(}\SpecialStringTok{f"Problema (a): t = }\SpecialCharTok{\{}\NormalTok{t}\SpecialCharTok{:.2f\}}\SpecialStringTok{, Interpolado = N/A, Real = }\SpecialCharTok{\{}\NormalTok{real}\SpecialCharTok{:.6f\}}\SpecialStringTok{, Error = N/A"}\NormalTok{)}
\end{Highlighting}
\end{Shaded}

\begin{verbatim}
[08-09 14:38:27][INFO] Resultados de interpolación y comparación con valores reales:
[08-09 14:38:27][INFO] Problema (a): t = 1.25, Interpolado = 1.014977, Real = 1.021957, Error = 0.006980
[08-09 14:38:27][INFO] Problema (a): t = 1.93, Interpolado = 1.153902, Real = 1.164390, Error = 0.010488
\end{verbatim}

\subsubsection{\texorpdfstring{b) \(y(t)=y(1.25)\) y
\(y(1.93)\)}{b) y(t)=y(1.25) y y(1.93)}}\label{b-yty1.25-y-y1.93}

\begin{Shaded}
\begin{Highlighting}[]
\NormalTok{a, b }\OperatorTok{=} \FloatTok{1.0}\NormalTok{, }\FloatTok{3.0}
\NormalTok{y\_t0 }\OperatorTok{=} \DecValTok{0}
\NormalTok{h }\OperatorTok{=} \FloatTok{0.2}
\NormalTok{f }\OperatorTok{=} \KeywordTok{lambda}\NormalTok{ t, y: }\DecValTok{1} \OperatorTok{+}\NormalTok{ y}\OperatorTok{/}\NormalTok{t}\OperatorTok{+}\NormalTok{(y}\OperatorTok{/}\NormalTok{t)}\OperatorTok{**}\DecValTok{2}
\NormalTok{ys\_b, ts\_b }\OperatorTok{=}\NormalTok{ ODE\_euler(a}\OperatorTok{=}\NormalTok{a, b}\OperatorTok{=}\NormalTok{b, f}\OperatorTok{=}\NormalTok{f, y\_t0}\OperatorTok{=}\NormalTok{y\_t0, h}\OperatorTok{=}\NormalTok{h)}
\NormalTok{t\_values\_b }\OperatorTok{=}\NormalTok{ [}\FloatTok{1.25}\NormalTok{, }\FloatTok{1.93}\NormalTok{]}
\NormalTok{results\_b }\OperatorTok{=}\NormalTok{ interpolate\_and\_compare(ts\_b, ys\_b, t\_values\_b, real\_b)}

\NormalTok{logging.info(}\StringTok{"Resultados de interpolación y comparación con valores reales:"}\NormalTok{)}

\ControlFlowTok{for}\NormalTok{ t, interpolated, real, error }\KeywordTok{in}\NormalTok{ results\_b:}
    \ControlFlowTok{if}\NormalTok{ interpolated }\KeywordTok{is} \KeywordTok{not} \VariableTok{None}\NormalTok{:}
\NormalTok{        logging.info(}\SpecialStringTok{f"Problema (b): t = }\SpecialCharTok{\{}\NormalTok{t}\SpecialCharTok{:.2f\}}\SpecialStringTok{, Interpolado = }\SpecialCharTok{\{}\NormalTok{interpolated}\SpecialCharTok{:.6f\}}\SpecialStringTok{, Real = }\SpecialCharTok{\{}\NormalTok{real}\SpecialCharTok{:.6f\}}\SpecialStringTok{, Error = }\SpecialCharTok{\{}\NormalTok{error}\SpecialCharTok{:.6f\}}\SpecialStringTok{"}\NormalTok{)}
    \ControlFlowTok{else}\NormalTok{:}
\NormalTok{        logging.info(}\SpecialStringTok{f"Problema (b): t = }\SpecialCharTok{\{}\NormalTok{t}\SpecialCharTok{:.2f\}}\SpecialStringTok{, Interpolado = N/A, Real = }\SpecialCharTok{\{}\NormalTok{real}\SpecialCharTok{:.6f\}}\SpecialStringTok{, Error = N/A"}\NormalTok{)}
\end{Highlighting}
\end{Shaded}

\begin{verbatim}
[08-09 14:39:50][INFO] Resultados de interpolación y comparación con valores reales:
[08-09 14:39:50][INFO] Problema (b): t = 1.25, Interpolado = 0.259722, Real = 0.283653, Error = 0.023931
[08-09 14:39:50][INFO] Problema (b): t = 1.93, Interpolado = 1.302427, Real = 1.490228, Error = 0.187801
\end{verbatim}

\subsubsection{\texorpdfstring{c) \(y(2.10)\) y
\(y(2.75)\)}{c) y(2.10) y y(2.75)}}\label{c-y2.10-y-y2.75}

\begin{Shaded}
\begin{Highlighting}[]
\NormalTok{a, b }\OperatorTok{=} \DecValTok{0}\NormalTok{, }\FloatTok{2.0}
\NormalTok{y\_t0 }\OperatorTok{=} \OperatorTok{{-}}\FloatTok{2.0}
\NormalTok{h }\OperatorTok{=} \FloatTok{0.2}
\NormalTok{f }\OperatorTok{=} \KeywordTok{lambda}\NormalTok{ t, y: }\OperatorTok{{-}}\NormalTok{(y }\OperatorTok{+} \DecValTok{1}\NormalTok{)}\OperatorTok{*}\NormalTok{(y }\OperatorTok{+} \DecValTok{3}\NormalTok{)}
\NormalTok{ys\_c, ts\_c }\OperatorTok{=}\NormalTok{ ODE\_euler(a}\OperatorTok{=}\NormalTok{a, b}\OperatorTok{=}\NormalTok{b, f}\OperatorTok{=}\NormalTok{f, y\_t0}\OperatorTok{=}\NormalTok{y\_t0, h}\OperatorTok{=}\NormalTok{h)}
\NormalTok{t\_values\_c }\OperatorTok{=}\NormalTok{ [}\FloatTok{1.3}\NormalTok{, }\FloatTok{1.93}\NormalTok{]}
\NormalTok{results\_c }\OperatorTok{=}\NormalTok{ interpolate\_and\_compare(ts\_c, ys\_c, t\_values\_c, real\_c)}
\NormalTok{logging.info(}\StringTok{"Resultados de interpolación y comparación con valores reales:"}\NormalTok{)}
\ControlFlowTok{for}\NormalTok{ t, interpolated, real, error }\KeywordTok{in}\NormalTok{ results\_c:}
    \ControlFlowTok{if}\NormalTok{ interpolated }\KeywordTok{is} \KeywordTok{not} \VariableTok{None}\NormalTok{:}
\NormalTok{        logging.info(}\SpecialStringTok{f"Problema (c): t = }\SpecialCharTok{\{}\NormalTok{t}\SpecialCharTok{:.2f\}}\SpecialStringTok{, Interpolado = }\SpecialCharTok{\{}\NormalTok{interpolated}\SpecialCharTok{:.6f\}}\SpecialStringTok{, Real = }\SpecialCharTok{\{}\NormalTok{real}\SpecialCharTok{:.6f\}}\SpecialStringTok{, Error = }\SpecialCharTok{\{}\NormalTok{error}\SpecialCharTok{:.6f\}}\SpecialStringTok{"}\NormalTok{)}
    \ControlFlowTok{else}\NormalTok{:}
\NormalTok{        logging.info(}\SpecialStringTok{f"Problema (c): t = }\SpecialCharTok{\{}\NormalTok{t}\SpecialCharTok{:.2f\}}\SpecialStringTok{, Interpolado = N/A, Real = }\SpecialCharTok{\{}\NormalTok{real}\SpecialCharTok{:.6f\}}\SpecialStringTok{, Error = N/A"}\NormalTok{)}
\end{Highlighting}
\end{Shaded}

\begin{verbatim}
[08-09 14:41:02][INFO] Resultados de interpolación y comparación con valores reales:
[08-09 14:41:02][INFO] Problema (c): t = 1.30, Interpolado = -1.103618, Real = -1.138277, Error = 0.034659
[08-09 14:41:02][INFO] Problema (c): t = 1.93, Interpolado = -1.022283, Real = -1.041267, Error = 0.018984
\end{verbatim}

\subsubsection{\texorpdfstring{d) \(y(t)=y(0.54)\) y
\(y(0.94)\)}{d) y(t)=y(0.54) y y(0.94)}}\label{d-yty0.54-y-y0.94}

\begin{Shaded}
\begin{Highlighting}[]
\NormalTok{a, b }\OperatorTok{=} \FloatTok{0.0}\NormalTok{, }\FloatTok{1.0}
\NormalTok{y\_t0 }\OperatorTok{=} \DecValTok{1}\OperatorTok{/}\DecValTok{3}
\NormalTok{h }\OperatorTok{=} \FloatTok{0.1}
\NormalTok{f }\OperatorTok{=} \KeywordTok{lambda}\NormalTok{ t, y: }\OperatorTok{{-}}\DecValTok{5}\OperatorTok{*}\NormalTok{y}\OperatorTok{+}\DecValTok{5}\OperatorTok{*}\NormalTok{t}\OperatorTok{**}\DecValTok{2}\OperatorTok{+}\DecValTok{2}\OperatorTok{*}\NormalTok{t}
\NormalTok{ys\_d, ts\_d }\OperatorTok{=}\NormalTok{ ODE\_euler(a}\OperatorTok{=}\NormalTok{a, b}\OperatorTok{=}\NormalTok{b, f}\OperatorTok{=}\NormalTok{f, y\_t0}\OperatorTok{=}\NormalTok{y\_t0, h}\OperatorTok{=}\NormalTok{h)}
\NormalTok{t\_values\_d }\OperatorTok{=}\NormalTok{ [}\FloatTok{0.54}\NormalTok{, }\FloatTok{0.94}\NormalTok{]}
\NormalTok{results\_d }\OperatorTok{=}\NormalTok{ interpolate\_and\_compare(ts\_d, ys\_d, t\_values\_d, real\_d)}
\NormalTok{logging.info(}\StringTok{"Resultados de interpolación y comparación con valores reales:"}\NormalTok{)}
\ControlFlowTok{for}\NormalTok{ t, interpolated, real, error }\KeywordTok{in}\NormalTok{ results\_d:}
    \ControlFlowTok{if}\NormalTok{ interpolated }\KeywordTok{is} \KeywordTok{not} \VariableTok{None}\NormalTok{:}
\NormalTok{        logging.info(}\SpecialStringTok{f"Problema (d): t = }\SpecialCharTok{\{}\NormalTok{t}\SpecialCharTok{:.2f\}}\SpecialStringTok{, Interpolado = }\SpecialCharTok{\{}\NormalTok{interpolated}\SpecialCharTok{:.6f\}}\SpecialStringTok{, Real = }\SpecialCharTok{\{}\NormalTok{real}\SpecialCharTok{:.6f\}}\SpecialStringTok{, Error = }\SpecialCharTok{\{}\NormalTok{error}\SpecialCharTok{:.6f\}}\SpecialStringTok{"}\NormalTok{)}
    \ControlFlowTok{else}\NormalTok{:}
\NormalTok{        logging.info(}\SpecialStringTok{f"Problema (d): t = }\SpecialCharTok{\{}\NormalTok{t}\SpecialCharTok{:.2f\}}\SpecialStringTok{, Interpolado = N/A, Real = }\SpecialCharTok{\{}\NormalTok{real}\SpecialCharTok{:.6f\}}\SpecialStringTok{, Error = N/A"}\NormalTok{)}
\end{Highlighting}
\end{Shaded}

\begin{verbatim}
[08-09 14:42:49][INFO] Resultados de interpolación y comparación con valores reales:
[08-09 14:42:49][INFO] Problema (d): t = 0.54, Interpolado = 0.282833, Real = 0.314002, Error = 0.031169
[08-09 14:42:49][INFO] Problema (d): t = 0.94, Interpolado = 0.866552, Real = 0.886632, Error = 0.020080
\end{verbatim}

\subsection{6. Use el método de Taylor de orden 2 para aproximar las
soluciones para cada uno de los siguientes problemas de valor
inicial.}\label{use-el-muxe9todo-de-taylor-de-orden-2-para-aproximar-las-soluciones-para-cada-uno-de-los-siguientes-problemas-de-valor-inicial.}

\begin{Shaded}
\begin{Highlighting}[]
\KeywordTok{def}\NormalTok{ ODE\_euler\_nth(}
    \OperatorTok{*}\NormalTok{,}
\NormalTok{    a: }\BuiltInTok{float}\NormalTok{,}
\NormalTok{    b: }\BuiltInTok{float}\NormalTok{,}
\NormalTok{    f: Callable[[}\BuiltInTok{float}\NormalTok{, }\BuiltInTok{float}\NormalTok{], }\BuiltInTok{float}\NormalTok{],}
\NormalTok{    f\_derivatives: List[Callable[[}\BuiltInTok{float}\NormalTok{, }\BuiltInTok{float}\NormalTok{], }\BuiltInTok{float}\NormalTok{]],}
\NormalTok{    y\_t0: }\BuiltInTok{float}\NormalTok{,}
\NormalTok{    N: }\BuiltInTok{int}
\NormalTok{) }\OperatorTok{{-}\textgreater{}} \BuiltInTok{tuple}\NormalTok{[}\BuiltInTok{list}\NormalTok{[}\BuiltInTok{float}\NormalTok{], }\BuiltInTok{list}\NormalTok{[}\BuiltInTok{float}\NormalTok{], }\BuiltInTok{float}\NormalTok{]:}
\NormalTok{    h }\OperatorTok{=}\NormalTok{ (b }\OperatorTok{{-}}\NormalTok{ a) }\OperatorTok{/}\NormalTok{ N}
\NormalTok{    t }\OperatorTok{=}\NormalTok{ a}
\NormalTok{    ts }\OperatorTok{=}\NormalTok{ [t]}
\NormalTok{    ys }\OperatorTok{=}\NormalTok{ [y\_t0]}

    \ControlFlowTok{for}\NormalTok{ \_ }\KeywordTok{in} \BuiltInTok{range}\NormalTok{(N):}
\NormalTok{        y }\OperatorTok{=}\NormalTok{ ys[}\OperatorTok{{-}}\DecValTok{1}\NormalTok{]}
\NormalTok{        T }\OperatorTok{=}\NormalTok{ f(t, y)}
\NormalTok{        ders }\OperatorTok{=}\NormalTok{ [}
\NormalTok{            h }\OperatorTok{/}\NormalTok{ factorial(m }\OperatorTok{+} \DecValTok{2}\NormalTok{) }\OperatorTok{*}\NormalTok{ mth\_derivative(t, y)}
            \ControlFlowTok{for}\NormalTok{ m, mth\_derivative }\KeywordTok{in} \BuiltInTok{enumerate}\NormalTok{(f\_derivatives)}
\NormalTok{        ]}
\NormalTok{        T }\OperatorTok{+=} \BuiltInTok{sum}\NormalTok{(ders)}
\NormalTok{        y }\OperatorTok{+=}\NormalTok{ h }\OperatorTok{*}\NormalTok{ T}
\NormalTok{        ys.append(y)}

\NormalTok{        t }\OperatorTok{+=}\NormalTok{ h}
\NormalTok{        ts.append(t)}
    \ControlFlowTok{return}\NormalTok{ ys, ts, h}
\end{Highlighting}
\end{Shaded}

\subsubsection{\texorpdfstring{a)\(y' = te^{3t}-2y, 0 \leq t \leq 1 ,  y(0) = 0\),
con \$ h = 0.5
\$}{a)y\textquotesingle{} = te\^{}\{3t\}-2y, 0 \textbackslash leq t \textbackslash leq 1 ,  y(0) = 0, con \$ h = 0.5 \$}}\label{ay-te3t-2y-0-leq-t-leq-1-y0-0-con-h-0.5}

\begin{Shaded}
\begin{Highlighting}[]
\NormalTok{f }\OperatorTok{=} \KeywordTok{lambda}\NormalTok{ t, y: t }\OperatorTok{*}\NormalTok{ exp(}\DecValTok{3} \OperatorTok{*}\NormalTok{ t) }\OperatorTok{{-}} \DecValTok{2} \OperatorTok{*}\NormalTok{ y}
\NormalTok{f\_p }\OperatorTok{=} \KeywordTok{lambda}\NormalTok{ t, y: exp(}\DecValTok{3} \OperatorTok{*}\NormalTok{ t) }\OperatorTok{*}\NormalTok{ (}\DecValTok{3} \OperatorTok{*}\NormalTok{ t }\OperatorTok{+} \DecValTok{1}\NormalTok{) }\OperatorTok{{-}} \DecValTok{2} \OperatorTok{*}\NormalTok{ (t }\OperatorTok{*}\NormalTok{ exp(}\DecValTok{3} \OperatorTok{*}\NormalTok{ t) }\OperatorTok{{-}} \DecValTok{2} \OperatorTok{*}\NormalTok{ y)}

\NormalTok{y\_t0 }\OperatorTok{=} \DecValTok{0}
\NormalTok{a }\OperatorTok{=} \DecValTok{0}
\NormalTok{b }\OperatorTok{=} \DecValTok{1}

\NormalTok{ys\_nth, ts\_nth, h }\OperatorTok{=}\NormalTok{ ODE\_euler\_nth(a}\OperatorTok{=}\NormalTok{a, b}\OperatorTok{=}\NormalTok{b, y\_t0}\OperatorTok{=}\NormalTok{y\_t0, f}\OperatorTok{=}\NormalTok{f, N}\OperatorTok{=}\DecValTok{2}\NormalTok{, f\_derivatives}\OperatorTok{=}\NormalTok{[f\_p])}
\BuiltInTok{print}\NormalTok{(}\StringTok{"Problema a:"}\NormalTok{)}
\BuiltInTok{print}\NormalTok{(}\SpecialStringTok{f"h = }\SpecialCharTok{\{}\NormalTok{h}\SpecialCharTok{\}}\SpecialStringTok{"}\NormalTok{)}
\BuiltInTok{print}\NormalTok{(}\SpecialStringTok{f"t: }\SpecialCharTok{\{}\NormalTok{ts\_nth}\SpecialCharTok{\}}\SpecialStringTok{"}\NormalTok{)}
\BuiltInTok{print}\NormalTok{(}\SpecialStringTok{f"y: }\SpecialCharTok{\{}\NormalTok{ys\_nth}\SpecialCharTok{\}}\SpecialStringTok{"}\NormalTok{)}
\BuiltInTok{print}\NormalTok{()}
\end{Highlighting}
\end{Shaded}

\begin{verbatim}
Problema a:
h = 0.5
t: [0, 0.5, 1.0]
y: [0, 0.125, 2.0232389682729033]
\end{verbatim}

\subsubsection{\texorpdfstring{b)
\(y' =1+(t-y)^2,2 \leq t \leq 3,  y(2) = 1\), con \$ h =
0.5\$}{b) y\textquotesingle{} =1+(t-y)\^{}2,2 \textbackslash leq t \textbackslash leq 3,  y(2) = 1, con \$ h = 0.5\$}}\label{b-y-1t-y22-leq-t-leq-3-y2-1-con-h-0.5}

\begin{Shaded}
\begin{Highlighting}[]
\NormalTok{f }\OperatorTok{=} \KeywordTok{lambda}\NormalTok{ t, y: }\DecValTok{1} \OperatorTok{+}\NormalTok{ (t }\OperatorTok{{-}}\NormalTok{ y) }\OperatorTok{**} \DecValTok{2}
\NormalTok{f\_p }\OperatorTok{=} \KeywordTok{lambda}\NormalTok{ t, y: }\OperatorTok{{-}}\DecValTok{2} \OperatorTok{*}\NormalTok{ (t }\OperatorTok{{-}}\NormalTok{ y) }\OperatorTok{*}\NormalTok{ (}\DecValTok{1} \OperatorTok{+}\NormalTok{ (t }\OperatorTok{{-}}\NormalTok{ y) }\OperatorTok{**} \DecValTok{2}\NormalTok{)}

\NormalTok{y\_t0 }\OperatorTok{=} \DecValTok{1}
\NormalTok{a }\OperatorTok{=} \DecValTok{2}
\NormalTok{b }\OperatorTok{=} \DecValTok{3}

\NormalTok{ys\_nth, ts\_nth, h }\OperatorTok{=}\NormalTok{ ODE\_euler\_nth(a}\OperatorTok{=}\NormalTok{a, b}\OperatorTok{=}\NormalTok{b, y\_t0}\OperatorTok{=}\NormalTok{y\_t0, f}\OperatorTok{=}\NormalTok{f, N}\OperatorTok{=}\DecValTok{2}\NormalTok{, f\_derivatives}\OperatorTok{=}\NormalTok{[f\_p])}
\BuiltInTok{print}\NormalTok{(}\StringTok{"Problema a:"}\NormalTok{)}
\BuiltInTok{print}\NormalTok{(}\SpecialStringTok{f"h = }\SpecialCharTok{\{}\NormalTok{h}\SpecialCharTok{\}}\SpecialStringTok{"}\NormalTok{)}
\BuiltInTok{print}\NormalTok{(}\SpecialStringTok{f"t: }\SpecialCharTok{\{}\NormalTok{ts\_nth}\SpecialCharTok{\}}\SpecialStringTok{"}\NormalTok{)}
\BuiltInTok{print}\NormalTok{(}\SpecialStringTok{f"y: }\SpecialCharTok{\{}\NormalTok{ys\_nth}\SpecialCharTok{\}}\SpecialStringTok{"}\NormalTok{)}
\BuiltInTok{print}\NormalTok{()}
\end{Highlighting}
\end{Shaded}

\begin{verbatim}
Problema a:
h = 0.5
t: [2, 2.5, 3.0]
y: [1, 1.5, 2.0]
\end{verbatim}

\subsubsection{\texorpdfstring{c)\(y' = 1+y/t, 1 \leq t \leq 2 ,  y(1) = 2\),
con \$ h =
0.25\$}{c)y\textquotesingle{} = 1+y/t, 1 \textbackslash leq t \textbackslash leq 2 ,  y(1) = 2, con \$ h = 0.25\$}}\label{cy-1yt-1-leq-t-leq-2-y1-2-con-h-0.25}

\begin{Shaded}
\begin{Highlighting}[]
\NormalTok{f }\OperatorTok{=} \KeywordTok{lambda}\NormalTok{ t, y: }\DecValTok{1} \OperatorTok{+}\NormalTok{ y }\OperatorTok{/}\NormalTok{ t}
\NormalTok{f\_p }\OperatorTok{=} \KeywordTok{lambda}\NormalTok{ t, y: }\OperatorTok{{-}}\NormalTok{y }\OperatorTok{/}\NormalTok{ t}\OperatorTok{**}\DecValTok{2} \OperatorTok{+} \DecValTok{1} \OperatorTok{/}\NormalTok{ t}

\NormalTok{y\_t0 }\OperatorTok{=} \DecValTok{2}
\NormalTok{a }\OperatorTok{=} \DecValTok{1}
\NormalTok{b }\OperatorTok{=} \DecValTok{2}

\NormalTok{ys\_nth, ts\_nth, h }\OperatorTok{=}\NormalTok{ ODE\_euler\_nth(a}\OperatorTok{=}\NormalTok{a, b}\OperatorTok{=}\NormalTok{b, y\_t0}\OperatorTok{=}\NormalTok{y\_t0, f}\OperatorTok{=}\NormalTok{f, N}\OperatorTok{=}\DecValTok{4}\NormalTok{, f\_derivatives}\OperatorTok{=}\NormalTok{[f\_p])}
\BuiltInTok{print}\NormalTok{(}\StringTok{"Problema a:"}\NormalTok{)}
\BuiltInTok{print}\NormalTok{(}\SpecialStringTok{f"h = }\SpecialCharTok{\{}\NormalTok{h}\SpecialCharTok{\}}\SpecialStringTok{"}\NormalTok{)}
\BuiltInTok{print}\NormalTok{(}\SpecialStringTok{f"t: }\SpecialCharTok{\{}\NormalTok{ts\_nth}\SpecialCharTok{\}}\SpecialStringTok{"}\NormalTok{)}
\BuiltInTok{print}\NormalTok{(}\SpecialStringTok{f"y: }\SpecialCharTok{\{}\NormalTok{ys\_nth}\SpecialCharTok{\}}\SpecialStringTok{"}\NormalTok{)}
\BuiltInTok{print}\NormalTok{()}
\end{Highlighting}
\end{Shaded}

\begin{verbatim}
Problema a:
h = 0.25
t: [1, 1.25, 1.5, 1.75, 2.0]
y: [2, 2.71875, 3.483125, 4.286102430555555, 5.122524181547619]
\end{verbatim}

\subsubsection{\texorpdfstring{d)\(y' = cos(2t)+sen(3t), 0 \leq t \leq 1 ,  y(0) = 1\),
con \$ h = 0.25
\$}{d)y\textquotesingle{} = cos(2t)+sen(3t), 0 \textbackslash leq t \textbackslash leq 1 ,  y(0) = 1, con \$ h = 0.25 \$}}\label{dy-cos2tsen3t-0-leq-t-leq-1-y0-1-con-h-0.25}

\begin{Shaded}
\begin{Highlighting}[]
\NormalTok{f }\OperatorTok{=} \KeywordTok{lambda}\NormalTok{ t, y: cos(}\DecValTok{2} \OperatorTok{*}\NormalTok{ t) }\OperatorTok{+}\NormalTok{ sin(}\DecValTok{3} \OperatorTok{*}\NormalTok{ t)}
\NormalTok{f\_p }\OperatorTok{=} \KeywordTok{lambda}\NormalTok{ t, y: }\OperatorTok{{-}}\DecValTok{2} \OperatorTok{*}\NormalTok{ sin(}\DecValTok{2} \OperatorTok{*}\NormalTok{ t) }\OperatorTok{+} \DecValTok{3} \OperatorTok{*}\NormalTok{ cos(}\DecValTok{3} \OperatorTok{*}\NormalTok{ t)}

\NormalTok{y\_t0 }\OperatorTok{=} \DecValTok{1}
\NormalTok{a }\OperatorTok{=} \DecValTok{0}
\NormalTok{b }\OperatorTok{=} \DecValTok{1}

\NormalTok{ys\_nth, ts\_nth, h }\OperatorTok{=}\NormalTok{ ODE\_euler\_nth(a}\OperatorTok{=}\NormalTok{a, b}\OperatorTok{=}\NormalTok{b, y\_t0}\OperatorTok{=}\NormalTok{y\_t0, f}\OperatorTok{=}\NormalTok{f, N}\OperatorTok{=}\DecValTok{4}\NormalTok{, f\_derivatives}\OperatorTok{=}\NormalTok{[f\_p])}
\BuiltInTok{print}\NormalTok{(}\StringTok{"Problema a:"}\NormalTok{)}
\BuiltInTok{print}\NormalTok{(}\SpecialStringTok{f"h = }\SpecialCharTok{\{}\NormalTok{h}\SpecialCharTok{\}}\SpecialStringTok{"}\NormalTok{)}
\BuiltInTok{print}\NormalTok{(}\SpecialStringTok{f"t: }\SpecialCharTok{\{}\NormalTok{ts\_nth}\SpecialCharTok{\}}\SpecialStringTok{"}\NormalTok{)}
\BuiltInTok{print}\NormalTok{(}\SpecialStringTok{f"y: }\SpecialCharTok{\{}\NormalTok{ys\_nth}\SpecialCharTok{\}}\SpecialStringTok{"}\NormalTok{)}
\BuiltInTok{print}\NormalTok{()}
\end{Highlighting}
\end{Shaded}

\begin{verbatim}
Problema a:
h = 0.25
t: [0, 0.25, 0.5, 0.75, 1.0]
y: [1, 1.34375, 1.7721870657725847, 2.110676064996487, 2.201643950842383]
\end{verbatim}

\subsection{7. Repita el ejercicio 6 con el método de Taylor de orden
4.}\label{repita-el-ejercicio-6-con-el-muxe9todo-de-taylor-de-orden-4.}

\begin{Shaded}
\begin{Highlighting}[]
\ImportTok{from}\NormalTok{ math }\ImportTok{import}\NormalTok{ exp, cos, sin, factorial}
\ImportTok{from}\NormalTok{ typing }\ImportTok{import}\NormalTok{ Callable, List}

\CommentTok{\# Definir las funciones y derivadas para cada problema}

\CommentTok{\# a) y\textquotesingle{} = t * e\^{}(3t) {-} 2y, y(0) = 0, h = 0.5}
\KeywordTok{def}\NormalTok{ f\_a(t: }\BuiltInTok{float}\NormalTok{, y: }\BuiltInTok{float}\NormalTok{) }\OperatorTok{{-}\textgreater{}} \BuiltInTok{float}\NormalTok{:}
    \ControlFlowTok{return}\NormalTok{ t }\OperatorTok{*}\NormalTok{ exp(}\DecValTok{3} \OperatorTok{*}\NormalTok{ t) }\OperatorTok{{-}} \DecValTok{2} \OperatorTok{*}\NormalTok{ y}

\KeywordTok{def}\NormalTok{ df\_a(t: }\BuiltInTok{float}\NormalTok{, y: }\BuiltInTok{float}\NormalTok{) }\OperatorTok{{-}\textgreater{}} \BuiltInTok{float}\NormalTok{:}
    \ControlFlowTok{return}\NormalTok{ exp(}\DecValTok{3} \OperatorTok{*}\NormalTok{ t) }\OperatorTok{*}\NormalTok{ (}\DecValTok{3} \OperatorTok{*}\NormalTok{ t }\OperatorTok{+} \DecValTok{1}\NormalTok{) }\OperatorTok{{-}} \DecValTok{2} \OperatorTok{*}\NormalTok{ (t }\OperatorTok{*}\NormalTok{ exp(}\DecValTok{3} \OperatorTok{*}\NormalTok{ t) }\OperatorTok{{-}} \DecValTok{2} \OperatorTok{*}\NormalTok{ y)}

\KeywordTok{def}\NormalTok{ ddf\_a(t: }\BuiltInTok{float}\NormalTok{, y: }\BuiltInTok{float}\NormalTok{) }\OperatorTok{{-}\textgreater{}} \BuiltInTok{float}\NormalTok{:}
    \ControlFlowTok{return} \DecValTok{3} \OperatorTok{*}\NormalTok{ exp(}\DecValTok{3} \OperatorTok{*}\NormalTok{ t) }\OperatorTok{*}\NormalTok{ (}\DecValTok{3} \OperatorTok{*}\NormalTok{ t }\OperatorTok{+} \DecValTok{2}\NormalTok{) }\OperatorTok{{-}} \DecValTok{2} \OperatorTok{*}\NormalTok{ (exp(}\DecValTok{3} \OperatorTok{*}\NormalTok{ t) }\OperatorTok{*}\NormalTok{ (}\DecValTok{3} \OperatorTok{*}\NormalTok{ t }\OperatorTok{+} \DecValTok{1}\NormalTok{))}

\KeywordTok{def}\NormalTok{ dddf\_a(t: }\BuiltInTok{float}\NormalTok{, y: }\BuiltInTok{float}\NormalTok{) }\OperatorTok{{-}\textgreater{}} \BuiltInTok{float}\NormalTok{:}
    \ControlFlowTok{return} \DecValTok{9} \OperatorTok{*}\NormalTok{ exp(}\DecValTok{3} \OperatorTok{*}\NormalTok{ t) }\OperatorTok{*}\NormalTok{ (}\DecValTok{3} \OperatorTok{*}\NormalTok{ t }\OperatorTok{+} \DecValTok{3}\NormalTok{) }\OperatorTok{{-}} \DecValTok{6} \OperatorTok{*}\NormalTok{ (}\DecValTok{3} \OperatorTok{*}\NormalTok{ exp(}\DecValTok{3} \OperatorTok{*}\NormalTok{ t) }\OperatorTok{*}\NormalTok{ (}\DecValTok{3} \OperatorTok{*}\NormalTok{ t }\OperatorTok{+} \DecValTok{2}\NormalTok{))}


\CommentTok{\# b) y\textquotesingle{} = 1 + (t {-} y)\^{}2, y(2) = 1, h = 0.5}
\KeywordTok{def}\NormalTok{ f\_b(t: }\BuiltInTok{float}\NormalTok{, y: }\BuiltInTok{float}\NormalTok{) }\OperatorTok{{-}\textgreater{}} \BuiltInTok{float}\NormalTok{:}
    \ControlFlowTok{return} \DecValTok{1} \OperatorTok{+}\NormalTok{ (t }\OperatorTok{{-}}\NormalTok{ y) }\OperatorTok{**} \DecValTok{2}

\KeywordTok{def}\NormalTok{ df\_b(t: }\BuiltInTok{float}\NormalTok{, y: }\BuiltInTok{float}\NormalTok{) }\OperatorTok{{-}\textgreater{}} \BuiltInTok{float}\NormalTok{:}
    \ControlFlowTok{return} \OperatorTok{{-}}\DecValTok{2} \OperatorTok{*}\NormalTok{ (t }\OperatorTok{{-}}\NormalTok{ y) }\OperatorTok{*}\NormalTok{ (}\DecValTok{1} \OperatorTok{+}\NormalTok{ (t }\OperatorTok{{-}}\NormalTok{ y) }\OperatorTok{**} \DecValTok{2}\NormalTok{)}

\KeywordTok{def}\NormalTok{ ddf\_b(t: }\BuiltInTok{float}\NormalTok{, y: }\BuiltInTok{float}\NormalTok{) }\OperatorTok{{-}\textgreater{}} \BuiltInTok{float}\NormalTok{:}
    \ControlFlowTok{return} \DecValTok{2} \OperatorTok{*}\NormalTok{ (}\DecValTok{1} \OperatorTok{+}\NormalTok{ (t }\OperatorTok{{-}}\NormalTok{ y) }\OperatorTok{**} \DecValTok{2}\NormalTok{) }\OperatorTok{{-}} \DecValTok{2} \OperatorTok{*}\NormalTok{ (t }\OperatorTok{{-}}\NormalTok{ y) }\OperatorTok{*}\NormalTok{ (}\OperatorTok{{-}}\DecValTok{2} \OperatorTok{*}\NormalTok{ (t }\OperatorTok{{-}}\NormalTok{ y))}

\KeywordTok{def}\NormalTok{ dddf\_b(t: }\BuiltInTok{float}\NormalTok{, y: }\BuiltInTok{float}\NormalTok{) }\OperatorTok{{-}\textgreater{}} \BuiltInTok{float}\NormalTok{:}
    \ControlFlowTok{return} \DecValTok{12} \OperatorTok{*}\NormalTok{ (t }\OperatorTok{{-}}\NormalTok{ y) }\OperatorTok{*}\NormalTok{ (}\OperatorTok{{-}}\DecValTok{2} \OperatorTok{*}\NormalTok{ (t }\OperatorTok{{-}}\NormalTok{ y)) }\OperatorTok{+} \DecValTok{8} \OperatorTok{*}\NormalTok{ (t }\OperatorTok{{-}}\NormalTok{ y) }\OperatorTok{*}\NormalTok{ (}\DecValTok{1} \OperatorTok{+}\NormalTok{ (t }\OperatorTok{{-}}\NormalTok{ y) }\OperatorTok{**} \DecValTok{2}\NormalTok{)}


\CommentTok{\# c) y\textquotesingle{} = 1 + y / t, y(1) = 2, h = 0.25}
\KeywordTok{def}\NormalTok{ f\_c(t: }\BuiltInTok{float}\NormalTok{, y: }\BuiltInTok{float}\NormalTok{) }\OperatorTok{{-}\textgreater{}} \BuiltInTok{float}\NormalTok{:}
    \ControlFlowTok{return} \DecValTok{1} \OperatorTok{+}\NormalTok{ y }\OperatorTok{/}\NormalTok{ t}

\KeywordTok{def}\NormalTok{ df\_c(t: }\BuiltInTok{float}\NormalTok{, y: }\BuiltInTok{float}\NormalTok{) }\OperatorTok{{-}\textgreater{}} \BuiltInTok{float}\NormalTok{:}
    \ControlFlowTok{return} \OperatorTok{{-}}\NormalTok{y }\OperatorTok{/}\NormalTok{ t}\OperatorTok{**}\DecValTok{2} \OperatorTok{+} \DecValTok{1} \OperatorTok{/}\NormalTok{ t}

\KeywordTok{def}\NormalTok{ ddf\_c(t: }\BuiltInTok{float}\NormalTok{, y: }\BuiltInTok{float}\NormalTok{) }\OperatorTok{{-}\textgreater{}} \BuiltInTok{float}\NormalTok{:}
    \ControlFlowTok{return} \DecValTok{2} \OperatorTok{*}\NormalTok{ y }\OperatorTok{/}\NormalTok{ t}\OperatorTok{**}\DecValTok{3} \OperatorTok{{-}} \DecValTok{1} \OperatorTok{/}\NormalTok{ t}\OperatorTok{**}\DecValTok{2}

\KeywordTok{def}\NormalTok{ dddf\_c(t: }\BuiltInTok{float}\NormalTok{, y: }\BuiltInTok{float}\NormalTok{) }\OperatorTok{{-}\textgreater{}} \BuiltInTok{float}\NormalTok{:}
    \ControlFlowTok{return} \OperatorTok{{-}}\DecValTok{6} \OperatorTok{*}\NormalTok{ y }\OperatorTok{/}\NormalTok{ t}\OperatorTok{**}\DecValTok{4} \OperatorTok{+} \DecValTok{2} \OperatorTok{/}\NormalTok{ t}\OperatorTok{**}\DecValTok{3}


\CommentTok{\# d) y\textquotesingle{} = cos(2t) + sin(3t), y(0) = 1, h = 0.25}
\KeywordTok{def}\NormalTok{ f\_d(t: }\BuiltInTok{float}\NormalTok{, y: }\BuiltInTok{float}\NormalTok{) }\OperatorTok{{-}\textgreater{}} \BuiltInTok{float}\NormalTok{:}
    \ControlFlowTok{return}\NormalTok{ cos(}\DecValTok{2} \OperatorTok{*}\NormalTok{ t) }\OperatorTok{+}\NormalTok{ sin(}\DecValTok{3} \OperatorTok{*}\NormalTok{ t)}

\KeywordTok{def}\NormalTok{ df\_d(t: }\BuiltInTok{float}\NormalTok{, y: }\BuiltInTok{float}\NormalTok{) }\OperatorTok{{-}\textgreater{}} \BuiltInTok{float}\NormalTok{:}
    \ControlFlowTok{return} \OperatorTok{{-}}\DecValTok{2} \OperatorTok{*}\NormalTok{ sin(}\DecValTok{2} \OperatorTok{*}\NormalTok{ t) }\OperatorTok{+} \DecValTok{3} \OperatorTok{*}\NormalTok{ cos(}\DecValTok{3} \OperatorTok{*}\NormalTok{ t)}

\KeywordTok{def}\NormalTok{ ddf\_d(t: }\BuiltInTok{float}\NormalTok{, y: }\BuiltInTok{float}\NormalTok{) }\OperatorTok{{-}\textgreater{}} \BuiltInTok{float}\NormalTok{:}
    \ControlFlowTok{return} \OperatorTok{{-}}\DecValTok{4} \OperatorTok{*}\NormalTok{ cos(}\DecValTok{2} \OperatorTok{*}\NormalTok{ t) }\OperatorTok{{-}} \DecValTok{9} \OperatorTok{*}\NormalTok{ sin(}\DecValTok{3} \OperatorTok{*}\NormalTok{ t)}

\KeywordTok{def}\NormalTok{ dddf\_d(t: }\BuiltInTok{float}\NormalTok{, y: }\BuiltInTok{float}\NormalTok{) }\OperatorTok{{-}\textgreater{}} \BuiltInTok{float}\NormalTok{:}
    \ControlFlowTok{return} \DecValTok{8} \OperatorTok{*}\NormalTok{ sin(}\DecValTok{2} \OperatorTok{*}\NormalTok{ t) }\OperatorTok{{-}} \DecValTok{27} \OperatorTok{*}\NormalTok{ cos(}\DecValTok{3} \OperatorTok{*}\NormalTok{ t)}


\CommentTok{\# Método de Taylor de orden 4}
\KeywordTok{def}\NormalTok{ ODE\_euler\_nth(}
    \OperatorTok{*}\NormalTok{,}
\NormalTok{    a: }\BuiltInTok{float}\NormalTok{,}
\NormalTok{    b: }\BuiltInTok{float}\NormalTok{,}
\NormalTok{    f: Callable[[}\BuiltInTok{float}\NormalTok{, }\BuiltInTok{float}\NormalTok{], }\BuiltInTok{float}\NormalTok{],}
\NormalTok{    f\_derivatives: List[Callable[[}\BuiltInTok{float}\NormalTok{, }\BuiltInTok{float}\NormalTok{], }\BuiltInTok{float}\NormalTok{]],}
\NormalTok{    y\_t0: }\BuiltInTok{float}\NormalTok{,}
\NormalTok{    N: }\BuiltInTok{int}
\NormalTok{) }\OperatorTok{{-}\textgreater{}} \BuiltInTok{tuple}\NormalTok{[}\BuiltInTok{list}\NormalTok{[}\BuiltInTok{float}\NormalTok{], }\BuiltInTok{list}\NormalTok{[}\BuiltInTok{float}\NormalTok{], }\BuiltInTok{float}\NormalTok{]:}
\NormalTok{    h }\OperatorTok{=}\NormalTok{ (b }\OperatorTok{{-}}\NormalTok{ a) }\OperatorTok{/}\NormalTok{ N}
\NormalTok{    t }\OperatorTok{=}\NormalTok{ a}
\NormalTok{    ts }\OperatorTok{=}\NormalTok{ [t]}
\NormalTok{    ys }\OperatorTok{=}\NormalTok{ [y\_t0]}

    \ControlFlowTok{for}\NormalTok{ \_ }\KeywordTok{in} \BuiltInTok{range}\NormalTok{(N):}
\NormalTok{        y }\OperatorTok{=}\NormalTok{ ys[}\OperatorTok{{-}}\DecValTok{1}\NormalTok{]}
\NormalTok{        T }\OperatorTok{=}\NormalTok{ f(t, y)}
\NormalTok{        ders }\OperatorTok{=}\NormalTok{ [}
\NormalTok{            h }\OperatorTok{**}\NormalTok{ (m }\OperatorTok{+} \DecValTok{1}\NormalTok{) }\OperatorTok{/}\NormalTok{ factorial(m }\OperatorTok{+} \DecValTok{2}\NormalTok{) }\OperatorTok{*}\NormalTok{ mth\_derivative(t, y)}
            \ControlFlowTok{for}\NormalTok{ m, mth\_derivative }\KeywordTok{in} \BuiltInTok{enumerate}\NormalTok{(f\_derivatives)}
\NormalTok{        ]}
\NormalTok{        T }\OperatorTok{+=} \BuiltInTok{sum}\NormalTok{(ders)}
\NormalTok{        y }\OperatorTok{+=}\NormalTok{ h }\OperatorTok{*}\NormalTok{ T}
\NormalTok{        ys.append(y)}

\NormalTok{        t }\OperatorTok{+=}\NormalTok{ h}
\NormalTok{        ts.append(t)}
    \ControlFlowTok{return}\NormalTok{ ys, ts, h}


\CommentTok{\# Definición de los problemas}
\NormalTok{problems }\OperatorTok{=}\NormalTok{ [}
\NormalTok{    \{}\StringTok{"a"}\NormalTok{: }\DecValTok{0}\NormalTok{, }\StringTok{"b"}\NormalTok{: }\DecValTok{1}\NormalTok{, }\StringTok{"f"}\NormalTok{: f\_a, }\StringTok{"f\_derivatives"}\NormalTok{: [df\_a, ddf\_a, dddf\_a], }\StringTok{"y\_t0"}\NormalTok{: }\DecValTok{0}\NormalTok{, }\StringTok{"N"}\NormalTok{: }\DecValTok{2}\NormalTok{\},   }\CommentTok{\# h = 0.5}
\NormalTok{    \{}\StringTok{"a"}\NormalTok{: }\DecValTok{2}\NormalTok{, }\StringTok{"b"}\NormalTok{: }\DecValTok{3}\NormalTok{, }\StringTok{"f"}\NormalTok{: f\_b, }\StringTok{"f\_derivatives"}\NormalTok{: [df\_b, ddf\_b, dddf\_b], }\StringTok{"y\_t0"}\NormalTok{: }\DecValTok{1}\NormalTok{, }\StringTok{"N"}\NormalTok{: }\DecValTok{2}\NormalTok{\},   }\CommentTok{\# h = 0.5}
\NormalTok{    \{}\StringTok{"a"}\NormalTok{: }\DecValTok{1}\NormalTok{, }\StringTok{"b"}\NormalTok{: }\DecValTok{2}\NormalTok{, }\StringTok{"f"}\NormalTok{: f\_c, }\StringTok{"f\_derivatives"}\NormalTok{: [df\_c, ddf\_c, dddf\_c], }\StringTok{"y\_t0"}\NormalTok{: }\DecValTok{2}\NormalTok{, }\StringTok{"N"}\NormalTok{: }\DecValTok{4}\NormalTok{\},   }\CommentTok{\# h = 0.25}
\NormalTok{    \{}\StringTok{"a"}\NormalTok{: }\DecValTok{0}\NormalTok{, }\StringTok{"b"}\NormalTok{: }\DecValTok{1}\NormalTok{, }\StringTok{"f"}\NormalTok{: f\_d, }\StringTok{"f\_derivatives"}\NormalTok{: [df\_d, ddf\_d, dddf\_d], }\StringTok{"y\_t0"}\NormalTok{: }\DecValTok{1}\NormalTok{, }\StringTok{"N"}\NormalTok{: }\DecValTok{4}\NormalTok{\},   }\CommentTok{\# h = 0.25}
\NormalTok{]}
\NormalTok{results }\OperatorTok{=}\NormalTok{ [ODE\_euler\_nth(}\OperatorTok{**}\NormalTok{problem) }\ControlFlowTok{for}\NormalTok{ problem }\KeywordTok{in}\NormalTok{ problems]}

\ControlFlowTok{for}\NormalTok{ i, (ys, ts, h) }\KeywordTok{in} \BuiltInTok{enumerate}\NormalTok{(results):}
    \BuiltInTok{print}\NormalTok{(}\SpecialStringTok{f"Problema }\SpecialCharTok{\{}\BuiltInTok{chr}\NormalTok{(}\DecValTok{97} \OperatorTok{+}\NormalTok{ i)}\SpecialCharTok{\}}\SpecialStringTok{:"}\NormalTok{)}
    \BuiltInTok{print}\NormalTok{(}\SpecialStringTok{f"h = }\SpecialCharTok{\{}\NormalTok{h}\SpecialCharTok{\}}\SpecialStringTok{"}\NormalTok{)}
    \BuiltInTok{print}\NormalTok{(}\SpecialStringTok{f"t: }\SpecialCharTok{\{}\NormalTok{ts}\SpecialCharTok{\}}\SpecialStringTok{"}\NormalTok{)}
    \BuiltInTok{print}\NormalTok{(}\SpecialStringTok{f"y: }\SpecialCharTok{\{}\NormalTok{ys}\SpecialCharTok{\}}\SpecialStringTok{"}\NormalTok{)}
    \BuiltInTok{print}\NormalTok{()}
\end{Highlighting}
\end{Shaded}

\begin{verbatim}
Problema a:
h = 0.5
t: [0, 0.5, 1.0]
y: [0, 0.18489583333333331, 2.3041147886173525]

Problema b:
h = 0.5
t: [2, 2.5, 3.0]
y: [1, 1.6458333333333333, 2.2593370602454668]

Problema c:
h = 0.25
t: [1, 1.25, 1.5, 1.75, 2.0]
y: [2, 2.7249348958333335, 3.4950996961805556, 4.303565100742335, 5.145247904523946]

Problema d:
h = 0.25
t: [0, 0.25, 0.5, 0.75, 1.0]
y: [1, 1.3289388020833333, 1.7296672968020275, 2.039934166759473, 2.1159884664152244]
\end{verbatim}




\end{document}
